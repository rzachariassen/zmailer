\begin{htmlonly}
\documentclass{article}
\usepackage{html,makeidx}
\setcounter{tocdepth}{6}
\setcounter{secnumdepth}{6}


%this font is used in  features.tex
\font\wncyr = wncyr10


\begin{htmlonly}
 \usepackage[dvips]{graphicx}
% \usepackage[dvips]{color}
% \usepackage[dvips,leftbars]{changebar}
% \newcommand{\FoilTeX}{\env{FoilTeX}}
%
 \def\url#1{\htmladdnormallink{#1}{#1}}
 \def\Email#1{\htmladdnormallink{<#1>}{mailto:#1}}
 \def\path#1{\texttt{#1}}
%% \def\urldef#1#2#3{\def#1{#2{#3}}}
 \def\glossary#1{\index{#1@\texttt{#1} \label{III#1}\htmlref{(G)}{GGG#1}}}
 \def\Glossary#1#2{\index{#1@{#2} \label{III#1}\htmlref{(G)}{GGG#1}}}
%
% hack to suppress changebar entries read from .aux file  !!! bug in changebar.perl !!! 
%% \makeatletter
%% \def\cb@barpoint#1#2#3{}
%% \makeatother
%
\newcommand{\sameas}[1]{\textcolor{red}{Same as setting: #1}}
%\newcommand{\onlinedocRM}{\url{http://www-math.mpce.mq.edu.au/\~{}ross/latex/manual/manual.html}}
%\newcommand{\EXcolors}{\url{http://www-math.mpce.mq.edu.au/\~{}ross/latex/crayola/crayola.html}}
%
\newcommand{\Lc}[1]{\texttt{\char92#1}} % LaTeX command
\newcommand{\Tc}[1]{\texttt{\char92#1}} % TeX command
\newcommand{\Cs}[1]{ \texttt{-#1} }   % command-line switch
\newcommand{\Ve}[1]{\index{#1@\texttt{#1}}\texttt{#1}} % version extension
\newcommand{\gsl}[1]{#1}
\newcommand{\indexentry}[2]{\item #1 #2}


\internal{}%
\internal{T}%
\internal{I}%
\internal{A}%
\internal{R}%
\internal{AP}%


\end{htmlonly}

%begin{latexonly}
\usepackage{array}

\newcommand{\Cs}[1]{{\upshape`\,\texttt{-#1}\,'}}
\newcommand{\Ve}[1]{\index{#1@\texttt{#1}}{\upshape`\,\texttt{#1}\,'}}
\newcommand{\Lc}[1]{{\upshape\ttfamily\char92#1}}
\newcommand{\Tc}[1]{{\upshape\ttfamily\char92#1}}

\def\mathsmiley{\smiley}
%end{latexonly}

% thanks to \KrisRose for this macro:
\def\smiley{\hbox{\rlap{$\bigcirc$}\kern1.3pt$\scriptstyle\ddot\smile$}}

\begin{imagesonly}
\def\mathsmiley{\smiley}
\end{imagesonly}


\newcommand{\Lcs}[1]{{\upshape\ttfamily\char92#1}}

\renewcommand{\thefootnote}{\arabic{footnote}}

%\newcommand{\godown}[1]{{\htmlref
% {\htmladdimg[left BORDER=0]{../psfiles/dn.gif}}{#1}}}
\newcommand{\godown}[1]{}

%\newcommand{\goback}[1]{{\htmlref
% {\htmladdimg[left BORDER=0]{../psfiles/up.gif}}{#1}}}
\newcommand{\goback}[1]{}

%\newcommand{\latextohtml}{\textup{\LaTeX 2{\ttfamily HTML}}}%
%\newcommand{\Perl}{\htmlref{\textsl{Perl}}{GGGPerl}}%  
%\newcommand{\PS}{\htmlref{\textup{Post\-Script}}{GGGpostscript}\Glossary{PostScript}{PostScript}{}}%
%\newcommand{\MF}{\htmlref{\textsl{Metafont}}{GGGMetafont}\Glossary{Metafont}{\textsl{Metafont}}{}}%
%\newcommand{\fn}[1]{\htmlref{\texttt{#1}}{GGG#1}\glossary{#1}}%  file names, with link to glossary
%\newcommand{\gn}[1]{\texttt{#1}\label{GGG#1}\htmlref{\^{}}{III#1}}%  file names, labelled within glossary
%%\newcommand{\appl}[1]{\htmlref{\textsl{#1}}{GGG#1}\Glossary{#1}{\textsl{#1}}{}}%  application software names
%\newcommand{\appl}[1]{\htmlref{\textsl{#1}}{GGG#1}\Glossary{#1}{\gsl{#1}}{}}%  application software names
%%
\newcommand{\env}[1]{{\upshape\sffamily #1}}%  LaTeX environment and package names
\newcommand{\HTMLtag}[1]{\path{<#1>}}%  HTML tag
\newcommand{\Meta}[1]{\texttt{\upshape<\textit{#1}>}}%  Meta string

\endinput

\section{Administration}
%% \chapter{Administration}

\begin{multicols}{2}

This section covers subsystem management issues, including
usage and configuration examples, basic and somewhat specific
explanations of how pre-existing scripts have been done.

\segment{zadm-dnsissues}{section}{DNS and ZMailer}
\segment{zadm-security}{section}{Security Issues}
\segment{zadm-queues}{section}{The Queue}
\clearpage
\segment{zadm-smtpserver}{section}{Smtpserver Configuration}
\clearpage
\segment{zadm-router}{section}{Router Configuration}
\clearpage
\segment{zadm-scheduler}{section}{Scheduler Configuration}
\clearpage
\segment{zadm-sm}{section}{{\em sm} Configuration}

\end{multicols}

\end{htmlonly}
\startdocument

This reference section tries to tell all the details of how each
subsystem parameter affects things, though perhaps without wide
explanations in every place. The usage examples and elaborations
are left (mostly) for the Administration part.

%begin{latexonly}
%
% ... some includes, like \segment{}{}{}s ?
%
%end{latexonly}

\subsection{DNS and ZMailer}

The cornerstone of everything in busy Internet email routing is a properly
working DNS server, and a modern resolver library. If you use the BIND 
nameserver, you should be using a recent version, at least BIND 4.8.
(As of this writing, bind is on version 8.1.1)

You can get improved DNS performance by installing local {\tt named},
which does cache replies, including {\bf negative} replies.

For the file {\tt /etc/resolv.conf}:
\begin{verbatim}
  domain  your.domain
  nameserver 127.0.0.1
  nameserver  (some other server)
\end{verbatim}


For the local nameserver daemon ({\em named\/}) you should have
at least following type of configuration:
\begin{verbatim}
  forwarders 10.12.34.56  10.45.67.89
  options forward-only
\end{verbatim}

which means that all the queries are attempted to be resolved
by the servers at IP addresses {\em 10.12.34.56\/} and
{\em 10.45.67.89\/}, and both the local server, and remote
servers will cache DNS responses.


\subsection{Security Issues}


Having local-parts that allow delivery to arbitary files, or which can trigger
execution of arbitrary programs, can clearly lead to a huge security
problem.  {\em sendmail\/} does address this problem, but in a restrictive and
unintuitive manner.  This aspect of ZMailer security has been designed to
allow the privileges expected by common sense.

The responsibility for implementing this kind of security is split between
the {\em router\/} and the transport agent that delivers a message to
an address.
Since it is the Transport Agent that must enforce the security, it needs
some information to guide it.  Specifically, for each address it delivers
to, some information about the ``trustworthyness'' of that address is
necessary so the transport agent can determine which privileges it can
assume when delivering for that destination.  This information is
determined by the {\em router\/}, and passed to the transport agent in the message
control file.  The specific measure of trustworthyness chosen by ZMailer, 
is simply a user id (uid) value representing the source of the address.

When a message comes in from a non-local host, the destination addresses
should obviously have no privileges on the local host (when mailing to a
file or a program).  Similarly, common sense would indicate that locally
originated mail should have the same privileges as the originator.  Based
on an initial user id assigned from such considerations, the privilege
attached to each address is modified by the attributes of the various alias
files that contain expansions of it.  The algorithm to determine the
appropriate privilege is to use the user id of the owner of the alias file
if and only if that file is not group or world writable, and the directory
containing the file is owned by the same user and is likewise neither group
nor world writable.  If any of these conditions do not hold, an
unprivileged user id will be assigned as the privilege level of the
address.



It is entirely up to the transport agent whether it will honour the
privilege assignment of an address, and indeed in many cases it might not
make sense (for example for outbound mail).  However, it is strongly
recommended that appropriate measures are taken when a transport agent has
no control over some action that may affect local files, security, or
resources.



The described algorithm is far from perfect. The obvious dangers are:

\begin{itemize}
\item The grandparent directories, to the Nth degree, are ignored, and may
not be secure. In that case all security is lost.
\item There is a window of vulnerability between when the permissions are
checked, and the delivery is actually made. This is the best argument
I have heard so far for embedding the local delivery program
(currently a separate transport agent) in the {\em router\/}.
\end{itemize}


There is also another kind of security that must be addressed.  That is the
mechanism by which the {\em router\/} is told about the origin of a message.  This
is something that must be possible for the message receiving programs
({\tt /bin/rmail} and the SMTP server are examples of these) to specify to
ZMailer.  The {\em router\/} knows of a list of trusted accounts on the system.  If
a message file is owned by one of these user id's, any sender specification
within the message file will be believed by ZMailer.  If the message file
is not owned by such a trusted account, the {\em router\/} will cross-check the
message file owner with any stated {\tt From:} or {\tt Sender:} address 
in the message header, or any origin specified in the envelope.  If a 
discrepancy is discovered, appropriate action will be taken.  This means 
that there is no way to forge the origin of a message without access to a 
trusted account.

Trusted accounts are those listed in the ZMailer group or the ``trusted'' 
variable in the system configuration ({\em router.cf\/}) file.



\subsection{The Queue}

There are multiple queues in ZMailer. Messages exist in
in one of five locations:

\begin{itemize}
\item Submission temporary directory ({\tt \$POSTOFFICE/public/})
\item Freezer directory ({\tt \$POSTOFFICE/freezer/})
\item Router directory ({\tt \$POSTOFFICE/router/})
\item Deferred directory ({\tt \$POSTOFFICE/deferred/})
\item Scheduler directories ({\tt \$POSTOFFICE/transport/}, 
{\tt \$POSTOFFICE/queue/})
\end{itemize}

and sometimes is also copied into the 
\begin{itemize}
\item Postmaster analysis area ({\tt \$POSTOFFICE/postman/})
\end{itemize}



\subsubsection{Message Submission Areas}



Message submission is done by writing a temporary file
into the directory ({\tt \$POSTOFFICE/public/}), the 
actual format of the submitted message is described in Appendix D.1.

Insert xref!

When the temporary file is completely written, it is renamed into one 
of the {\em router\/} input directories, usually into 
{\tt \$POSTOFFICE/router/} with a name that is a decimal representation
of the file i-node number. This is a way to ensure that the
name of the file in the {\tt \$POSTOFFICE/router/} directory is unique.

The message may also be renamed into alternate router directories, 
which give lower priorities on which messages to process when.

Sometimes, especially {\em smtpserver\/} built files may be moved
into alternate directories. The smtpserver {\tt ETRN}
command is implemented by moving the built special file to the 
directory {\tt \$POSTOFFICE/transport/} without going through the
{\em router\/}.  The smtpserver may also move newly arrived
files into the {\tt \$POSTOFFICE/freezer/} directory.




\subsubsection{Router Behaviour on Queues}



The {\em router\/} processes have a few different behaviours when
they go over their input directories.

First of all, if there are {\tt ROUTERDIRS} entries, those
are scanned for processing after the primary 
{\tt \$POSTOFFICE/router/} directory is found empty.

Within each directory, the {\em router\/} will sort files at first into 
mod-time order, and then process the oldest message first. (Unless the 
{\em router\/} has been started with the `-s' option.)

The {\em router\/} acquires a lock on the message (spool file) by means
of renaming the file from its previous name to a name with the format: 
{\tt concat(file\_ino\_number,"-",progpid)}

Once the {\em router\/} has been able to acquire a new name
for the file, it starts off by creating a temporary file of
{\em router\/} routing decisions.  The file has a name with the format:
{\tt concat("..",file\_ino\_number,"-",progpid)}

Once the processing has completed successfully, the original
input file is moved into the directory {\tt \$POSTOFFICE/queue/}, and
the {\em router\/} produced scheduler work-specification file is moved
to the {\tt \$POSTOFFICE/transports/} directory with the {\bf same}
name that the original file has.

If the routing analysis encountered problems, the message
may end up moved into the directory {\tt \$POSTOFFICE/deferred/}, from
which the command {\tt zmailer resubmit} is needed to return 
the messages to processing. (The {\em router\/} logs should be consulted 
for the reason why the message ended up in the {\em /deferred/\/} area,
especially if the command {\tt zmailer resubmit} is not able
to get the messages processed successfully and
the files end up back in the {\em /deferred/\/} area.)

If the original message had errors in its RFC-822 compliance,
or some other detail, a copy of the message may end up in the 
directory {\tt \$POSTOFFICE/postman/}.  

See 
section
\ref{postmaster_analysis_area}
, Postmaster Analysis Area, page 
\pageref{postmaster_analysis_area}.




\subsubsection{Scheduler and Transport Agents}



The scheduler work specification files are in the directory
{\tt \$POSTOFFICE/transport/}, under which there can be (optionally)
one or two levels of subdirectories into which the actual
work files will be scattered to lessen the sizes of individual
directories in which files reside, and thus to speed up the
system implied directory lookups at the transport agents, when
they open files, (and also in the scheduler).

When the {\em router\/} has completed message file processing, it places the
resulting files into the top level directory of the scheduler; 
{\tt \$POSTOFFICE/transport/}, and {\tt \$POSTOFFICE/queue/}.

The scheduler (if so configured) will move the messages into 
``hashed subdirectories,'' when it finds the work specification files, 
and then start processing addresses in them.

The transport agents are run with their CWD in directory 
{\tt \$POSTOFFICE/transport/}, 
and they open files relative to that location. Actual message bodies, when 
needed, are opened with the path prefix ``{\tt ../queue/}'' to the work 
specification file name.

Usually it is the transport agent's duty to log different permanent status reports 
(failures, successes) into the end of the work-specification file.  
Sometimes the scheduler also logs something at the end of this file.  
All such operations are attempted {\bf without} any sort 
of explicit locking, instead trusting the {\em write(2)\/} system call to 
behave in an atomic manner while writing to the disk file, and 
having a single buffer of data to write.

Once the scheduler has had all message recipient addresses 
processed by the transport agents, it will handle possible
diagnostics on the message, and finally it will remove the
original spool-file from the {\tt \$POSTOFFICE/queue/}, and 
the work-specification file from {\tt \$POSTOFFICE/transport/}.




\subsubsection{Postmaster Analysis Area\label{postmaster_analysis_area}}



If the filename in the {\tt \$POSTOFFICE/postman/} directory has an 
underscore in it, the reason for the copy is {\bf soft}, that is, the 
message has been sent through successfully in spite of being copied into
the {\tt \$POSTOFFICE/postman/} directory.

If the filename in the {\tt \$POSTOFFICE/postman/}
{\em does not have an underscrore in it, that file has not been processed
successfully, and the\/} {\bf only} {\em copy of the message
is now in {\tt \$POSTOFFICE/postman/}\/}

If the  smtpserver  receives a message with content that the policy filtering
system decides to be dubious, it can move the message into
{\tt \$POSTOFFICE/freezer/} directory with a bit explanatory name of type:
{\tt concat(file\_ino\_number,"explanatory-text")}.

The files in that area are such that they are input to the router, and
as of this wiring, there are no tools to automatically process them for
obvious spams, and just those were falsely triggered.




\subsection{\label{Logging_and_Statistics}Logging and Statistics }

This text has been copied from other sections and will change...



\subsubsection{Logging}


\subsubsection{Checking the Log Files}

Start ZMailer:
\begin{verbatim}
  $MAILBIN/zmailer
\end{verbatim}


Keep an eye on the log files ({\tt \$LOGDIR/\{router,scheduler\}}),
the {\tt \$POSTOFFICE/postman} directory for malformed message files,
and {\tt \$POSTOFFICE/deferred} in case of resource problems.


\subsubsection{Trimdown of Logging}

Once satisfied that things appear to work, you may want to trim down
logging: there are 4 kinds of logging to deal with:
\begin{itemize}
\item router logs, usually kept in {\tt \$LOGDIR/router}.  This is the stdout
and stderr output of the router daemon.  If you wish to turn it off,
invoke router with a {\tt -L/dev/null} option, i.e. change the zmailer
script.  Alternatively, modify the {\tt log()} function in the
configuration file, or its invocations.

*** NOTE, THIS IS INCORRECT INFO, see  \$MAILSHARE/cf/standard.cf for
*** routine   dribble(),  and especially its invocations!
\item scheduler logs, usually kept in {\tt \$LOGDIR/scheduler}.  Same as router.
\item general mail logs, usually kept in syslog files, depending on how
you have configured the syslog utility ({\tt /etc/syslog.conf}).  All
ZMailer programs log using the LOG\_MAIL facility code for normal
messages.  You can deal with this specifically in your {\tt syslog}
configuration file on systems with a 4.3bsd-based syslog.  The
following reflects the recommended configuration on SunOS 4.0:
\begin{verbatim}
mail.crit               /var/log/syslog
mail.debug              /var/log/mail
\end{verbatim}

For pre-4.3bsd-based syslogs, you may want the syslog log file
to be just for important messages (e.g. LOG\_NOTICE and higher
priority), and have a separate file for informational messages
(LOG\_DEBUG and up).
\item By default, the postmaster will receive a copy of all bounced
mail; this can be turned off selectively by simply editing the
various canned forms used to create the error messages.  These
forms are located in the FORMSDIR ({\tt proto/forms} in the distribution,
or {\tt \$MAILSHARE/forms} when installed).  You should review these
in any case to make sure the text is appropriate for your site.
\end{itemize}




\subsubsection{Scheduler Statistics log}

The  statistics log reports condenced performance oriented
information in following format:
\begin{verbatim}
timestamp fileid      dt1  dt2 state $channel/$host
812876190 90401-2       0    5    ok usenet/-
812876228 90401-1       0    7    ok usenet/-
812876244 90401-1       0    1    ok local/gopher-admin
812876244 90401-1       0    5    ok smtp/funet.fi
812876559 90401-1       0   21    ok smtp/utu.fi
\end{verbatim}

where the fields are:
\begin{description}
\item[timestamp] \mbox{}

The original spoolfile {\tt ctime} (creation time)  stamp
in decimal.

\item[fileid] \mbox{}
Spoolfile name after the router has processed it.

\item[dt1] \mbox{}
The time difference from spoolfile ctime to 
scheduler control file creation by the router.

\item[dt2] \mbox{}
The time difference from scheduler file {\tt ctime} to
the delivery that is logged on.

\item[state] \mbox{}
What happened?  Values: {\tt ok}, {\tt error}, {\tt expiry}

\item[\$channel-\$host] \mbox{}
Where/how it was processed.

\end{description}









\subsection{SMTP-Server Configuration}

{\tt MAILSHARE/smtpserver.conf}

The {\em smtpserver\/} program actually has several operational modes.
It can operate as a stand-alone internet service socket listener, which
forks off childs that do the actual SMTP-protocol service.

It can be started from under the control of the inetd server, and it can
there fulfill the same role as it does stand-alone mode.

It can even be used to accept Batch-SMTP from incoming files (UUCP,
and BITNET uses, for example).

Usage

{\tt smtpserver [ -46aignvBV ] [ -p port ] [ -l logfile ] [ -s [ftveR] ] [ -L maxloadaver ] [ -M SMTPmaxsize ] [ -P postoffice ] [ -R router ] [ -C cfgfile ]}

The commonly used command line options are:

{\tt smtpserver [-aBivn] [-s ehlo-styles] [-l logfile] [-C cfgpath]}

Without any arguments the  smtpserver  will start as a daemon
listening on TCP port 25 (SMTP).

The options are:

\begin{description}
\item[{\tt -a}] \mbox{}

Query IDENT information about the incoming connection. This information (if available at all) may, or may not tell who is forming a connection.

\item[{\tt -B}] \mbox{}

The session is Batch-SMTP a.k.a. BSMTP type of session. Use of "-i" option is needed, when feeding the input batch file.

\item[{\tt -i}] \mbox{}

This is interactive session.  IO is done thru stdin/stdout.

\item[{\tt -v}] \mbox{}

Verbose trace written to stdout for use in conjunktion with "-i", and "-B".

\item[{\tt -n}] \mbox{}

This tells that the smtpserver is running under inetd, and that the stdin/stdout file handles are actually network sockets on which we can do peer identity lookups, for example.

\item[{\tt -s ehlo-style}] \mbox{}

Default value for various checks done at MAIL FROM, RCPT TO, VRFY, and EXPN commands. These are overridden with the value from EHLO-patterns, if they are available (more below)

\item[{\tt -l logfilepath}] \mbox{}

Filename for the smtpserver input protocol log. This logs about everything, except actual message data...

\item[{\tt -C configfilepath}] \mbox{}

Full path for the smtpserver configuration in case the default location can not be used for some reason.

\item[{\tt -M}] \mbox{}

SMTPmaxsize Defines the asolute maximum size we accept from incoming email. (Default: infinite) (This is a local policy issue.)

\end{description}


{\bf Configuration}

If the system has file  {\tt \$MAILSHARE/smtpserver.conf} (by default),
that file is read for various parameters, which can override a few
of those possibly issued at the command line.

Example (shortened) configuration is:

%begin{latexonly}
\begin{small}
%end{latexonly}
\begin{verbatim}
#PARAM maxsize              10000000    # Same as -M -option
#PARAM max-error-recipients        3    # More than this is propably SPAM!
#PARAM MaxSameIpSource            10    # Max simultaneous connections from
#                                       # any IP source address
#PARAM ListenQueueSize             5    # The  listen(2)  parameter
#
PARAM help -------------------------------------------------------------
PARAM help  This mail-server is at Yoyodyne Propulsion Inc.
PARAM help  Our telephone number is: +1-234-567-8900, and
...
PARAM help -------------------------------------------------------------
#
PARAM   accept-percent-kludge # "localpart" can contain '%' and '!'
#PARAM  reject-percent-kludge # "localpart" can't contain  --"--
#
#  Uncomment this for not to strip source routes from:  <@a,@b:c@d>
#PARAM AllowSourceRoute
#
# The policy database:  (NOTE: See  'makendbm'  for its default suffixes!)
PARAM  policydb   @DBTYPE@  @MAILVAR@/db/smtp-policy
#
# HELO/EHLO-pattern     style-flags
#               [max loadavg]
#
localhost           999 ftveR
some.host.domain    999 !NO EMAIL ACCEPTED FROM YOUR MACHINE
\[*\]               999 ve
*                   999 veR
\end{verbatim}
%begin{latexonly}
\end{small}
%end{latexonly}





The PARAM keywords and values are:

\begin{description}
\item[{\tt maxsize nn}] \mbox{}

Maximum size in the number of bytes of the entire spool message
containing both the transport envelope, and the actual message.
That is, if the max-size is too low, and there are a lot of
addresses, the message may entirely become undeliverable..



\item[{\tt max-error-recipients nn}] \mbox{}

In case the message envelope is an error envelope (MAIL FROM:{\(<\)}{\(>\)}),
the don't accept more than this many separate recipient addresses
for it. The default value is 3, which should be enough for most cases.
(Some SPAMs claim to be error messages, and then provide a huge
number of recipient addresses...)



\item[{\tt MaxSameIpSource nn}] \mbox{}

(Effective only on daemon-mode server -- not on "-i", nor "-n" modes.)
Sometimes some systems set up multiple parallel connections to same
host  (qmail ones especially, not that ZMailer has entirely clean
papers on this - at least up to 2.99.X series), we won't accept
more than this many connections from the same IP source address
open in parallel.  The default value for this limit is 10.

The principal reason for this has been authors experience at nic.funet.fi,
where some MS-Windows users have caused {\bf huge} numbers
of parallel connections to some services.  So huge, that the system
did in fact run out of swap due to that, and caused all manner of
other nasty things to occur too...  All this simply because some
windows client had opened 800+ parallel sessions, and each server
process consumed separate blocks of swap space...  To avoid the
system to succumb under such an accidental denial-of-service attack
at the overall system, this parallel limit was created.



\item[{\tt ListenQueueSize nn}] \mbox{}

This relates on some newer systems where the {\tt listen(2)} system call
can define higher limits, than the traditional/original 5.  This
limit tells how many nascent TCP streams we can have in SYN\_RCVD
state before we stop answering to incoming SYN packets requesting
opening of a connection.

There are entirely deliberate denial-of-service attacks based on
flooding to some server many SYNS on which it can't send replies
back (because the target machines don't have network connectivity,
for example), and thus filling the back-queue of nascent sockets.
This can also happen accidentally, as the connectivity in between
the client host, and the server host may have a black hole into
which the SYN-ACK packets disappear, and the client thus will not
be able to get the TCP startup three-way handshake completed.

Some newer systems can have this value upped to thousands to
improve systems resiliency against malicious attacks, and most
likely to provide complete immunity against the accidental
"attack" by the failing network routing.



\item[{\tt help 'string'}] \mbox{}

This one adds yet another string (no quotes are used) into those
that are presented to the client when it asks for "HELP" in the
SMTP session.



\item[{\tt AllowSourceRoute}] \mbox{}

When this is present, possible incoming  {\(<\)}@aa,@bb:cc@dd{\(>\)}  type of
address is accepted in entirety without stripping it into {\(<\)}cc@dd{\(>\)}.
The default is that this is not present, and the RFC-821 source-
route address is stripped into its basic form.



\item[{\tt PolicyDB dbtype dbpath}] \mbox{}

This defines the database type, and file path prefix to the binary
database containing policy processing information.  More of this
below.  Actual binary database file names are formed by appending
type specific suffixes to the path prefix.  For example NDBM
database appends ".pag" and ".dir", while BSD-Btree appends only
".db".  (And the latter has only one file, while the first has two.)

\end{description}


All lines that are not comments, nor start with uppercase keyword
"POLICY" are "EHLO-style patterns".





Delete to end of section?

If the {\tt MAILSHARE/{\bf smtpserver.conf}} file exists it is read 
to configure two kinds of things:

\begin{itemize}
\item  PARAM -entries
\end{itemize}


Allow server start-time parametrization of several things, including:

\begin{itemize}
\item  system parameters
\item  help texts
\item  acceptance/rejection database definitions
\item The style ({\tt -s}) option
\end{itemize}


Behaviour is based on glob patterns matching the 
{\bf HELO/EHLO} name given by a remote client. Lines 
beginning with a \# or whitespace are ignored in the file, and 
all other lines must consist of two tokens: a shell-style (glob) 
pattern starting at the beginning of the line, whitespace, and a 
sequence of style flags. The first matching line is used. As a 
special case, the flags section may start with a ! character in 
which case the remainder of the line is a failure comment message 
to print at the client. This configuration capability is intended 
as a way to control misbehaving client software or mailers.






\subsubsection{Policy Based Filtering}



The {\tt policy-builder.sh} script bundles together a set of other
things before policy filter is ready for use:

\begin{verbatim}
        DB/smtp-policy.src                      The boilerplate
        DB/localnames           ('= _full_rights')
        DB/smtp-policy.relay    ('= _full_rights')
        DB/smtp-policy.mx       ('relaytargets +')
        DB/smtp-policy.spam     ('= _bulk_mail')
\end{verbatim}


At the moment, {\tt smtp-policy.spam} source is retrieved with LYNX from
the URL: {\tt http://www.webeasy.com:8080/spam/spam\_download\_table}
however it seems there are sites out there that are spam havens, and
that serve valid spam source/responce domains, which are not registered
at that database.

If you want, you can modify your boilerplate as well as your
installed {\tt policy-builder.sh} script. Doing {\tt make install} will
overwrite {\tt policy-builder.sh}, but not {\tt smtp-policy.src}.



Basically these various source files (if they exist) are used to
combine knowledge of valid users around us:

\begin{description}
\item[localnames] \mbox{}

Who we are -- ok for receiving ('\_full\_rights'
is a bit too broad for that, though... because
its value 'relaycustomer +' can allow anybody
to use us for relaying by claiming some local
identity at MAIL FROM:{\(<\)}..{\(>\)}.



\item[smtp-policy.relay] \mbox{}

Who can use us as outbound relay,
Use  $ip.number$/maskwidth  here for listing
those senders (networks) we absolutely trust.
We can set the "always\_accept" flag at the
source IP test, and never after.



\item[smtp-policy.mx] \mbox{}

Who really are our MX clients.
Use this when you really know them, and don't
want just to trust that if recipient has MX
to you, it would be ok...
You can substitute this knowledge with a fuzzy
feeling by using the `acceptifmx +' attribute in
the generic boilerplate.



\item[smtp-policy.spam] \mbox{}

Those users and domains that are
absolutely a no-no for senders, or recipients
no matter what earlier analysis has shown.
(Except for those that we absolutely trust..)

\end{description}





\subsection{Router Configuration}



The names (determined at compile-time) and interface specifications for the
routing and crossbar functions, are the only crucial ``magical'' things one
needs to contend with in a proper {\em router\/} configuration.  The syntax and
semantics of the configuration file's contents are dealt with in the
following subsection. The details of the two functions introduced here are
specified after that, once the necessary background information has been
given.

{\em Router\/} behavior is controlled by a configuration file read at 
startup. It is a {\tt zmsh(1)} script that uses facilities provided 
built into the {\em router\/}. 

The configuration file looks like a Bourne Shell script at first glance.
There are minor syntax changes from standard {\tt sh}, but the aim is to be as
close to the Bourne Shell language as is practical. The contents of the
file are compiled into a parse tree, which can then be interpreted by the
{\em router\/}.  The configuration file is usually self-contained, although an easy
mechanism exists to make use of external UNIX programs when so desired.
Together with a very flexible database lookup mechanism, functions, and
address manipulation based on token-matching regular expressions, the
configuration file language is an extremely flexible substrate to
accomplish its purpose. When the language is inadequate, or if speed
becomes an issue, it is possible to call built in (C coded) functions. The
interface to these functions is mostly identical to what a standalone
program would expect (modulo symbol name clashes and return values), to
ease migration of external programs to inclusion in the {\em router\/} process.




\subsubsection{Configuration File Programming Language}



Whenever the {\em router\/} process starts, its first action is to read its
configuration file.  The configuration file is a text file which contains
statements interpreted immediately when the file is read.  Some statements
are functions, in which case the function is defined at that point in
reading the configuration file.  The purpose of the configuration file is
to provide a simple way to customize the behavior of the mailer, and this
is primarily achieved by defining the {\tt router} and {\tt crossbar} 
functions. For these to work properly, some initialization code and auxiliary
functions will usually be needed.

At first sight, a configuration file looks like a Bourne shell script.
The ideal is to duplicate the functionality, syntax, and to a large
degree the semantics, of a shell script.  Therefore, the configuration file
programming language is defined in terms of its deviation from standard
Bourne shell syntax and semantics.  The present differences are:

\begin{itemize}
\item  No {\tt repeat} statement.
\item  Functions are allowed, parameter lists are allowed. If not enough
arguments are present in a function call to exhaust the parameter
list, the so-far unbound parameter variables are bound to `' (the
empty string) as local variables. For example, this is the identity
address rewriting function:
     
\begin{verbatim}
       null (address) {
          return $address         # surprise!
       }
\end{verbatim}

\item  Multiple-value returns are allowed.  The {\tt return} statement can be
used to return a non-`' value from a function.  The following are all
legal {\tt return} statements:
     
\begin{verbatim}
          return
          return $address
          return $channel ${next_host} ${next_address}
\end{verbatim}

\item  Variables are dynamically scoped, local variables are the 
ones in a function's parameter list and those declared with the 
``{\tt local}'' statement. Only the first value of a multiple-value return may 
be assigned to a variable.  All values are strings, so no type information, 
checking, or declaration, is  necessary.
\item  Quoting is a bit stilted. All quotes (double-, single-, back-), must 
appear in matching pairs at the beginning and end of a word.  Single 
quotes are not stripped, double quotes cause the enclosed character 
sequence to be collected into a quoted-string RFC822 token.  For 
example, the statement:

\begin{verbatim}
          foo `bar "`baz`"`
\end{verbatim}
     
is evaluated as {\tt (apply 'foo (apply 'bar (baz)))}.
\item  The forms {\tt \$\{variable:=value\}}, 
{\tt \$\{variable:-value\}}, and
{\tt \$\{variable:+value\}} are supported.  
\item  Patterns (in case labels) are evaluated once, the first time they are
encountered.
\item  At the end of a case label, the sequentially next case labels of the
same case statement will be tried for successful pattern matching (and
the corresponding case label body executed). The only exceptions
(apart from encountering a return statement) are:

{\tt again}     

a function which retries the current case label for a match.

{\tt break}     

continues execution after the current case statement.
\item  Various standard Bourne shell functions do not exist built in.
\item  The general form of function calls in the system is:
\$(funcname arguments)
It returns a scalar or list object, and the result can be stored
into variables at will.
\item  Relations, and other database lookups are contructed as function calls
where the relation name is the function name.  More about this later.
\end{itemize}


There are currently only three entry points (i.e. magic names known to the
{\em router\/} code) in the configuration file, namely the {\tt process}, {\tt router} and 
{\tt crossbar} functions.

The {\tt process} function is called with a file name as argument. 
The file is typically located in the {\tt \$POSTOFFICE/router/} 
directory. {\tt Process} is a protocol switch function which uses the 
form of the file name to determine how to process different types of 
messages.

The {\tt router} function is called with an address as argument, and returns a
triple of (channel, host, user) as three separate values, corresponding to
the channel the message should be sent out on (or, the router function can
also be called to check on who sent a message), the host or node name for
that channel (null if local delivery), and the address the receiving agent
should transmit to.

The {\tt crossbar} function is in charge of rewriting envelope addresses,
selecting message header address munging type (a function to be called with
each message header address), and possibly doing per-message logging or
enforcing restrictions deemed necessary. It takes a sender-triple and a
receiver-triple as arguments (six parameters altogether). It returns the
new values for each element of the two triples, and in addition a function
name corresponding to the function to be used to rewrite header addresses
for the specific destination.  If the destination is to be ignored,
returning a null function name will accomplish this.

The {\em router\/} has several built in (C coded) functions.  Their calling
sequence and interface specification is exactly the same as for the
functions defined in the configuration file.  Some of these functions have
special semantics, and they fall into three classes, as follows:

Functions that are critical to the proper functioning of the configuration
file interpreter:

\begin{description}
\item[{\tt return}] \mbox{}

returns its argument(s) as the value of a function call

\item[{\tt again}] \mbox{}

repeats the current case label

\item[{\tt break}] \mbox{}

exits a case statement

\end{description}


Functions that are necessary to complete the capabilities of the
interpreter:

\begin{description}
\item[{\tt relation}] \mbox{}

defines a database to the database lookup mechanism

\item[{\tt sh}] \mbox{}

an internal function which runs its arguments as {\tt /bin/sh} would

\end{description}


Non-critical but recommended functions:

\begin{description}
\item[{\tt getzenv}] \mbox{}

retrieves global ZMailer configuration values

\item[{\tt echo}] \mbox{}

emulates {\tt /bin/echo}

\item[{\tt exit}] \mbox{}

aborts the {\em router\/} with the specified status code

\item[{\tt hostname}] \mbox{}

internal function to get and set the system name

\item[{\tt trace}] \mbox{}

turns on selected debugging output

\item[{\tt untrace}] \mbox{}

turns off selected debugging output

\item[{\tt [}] \mbox{}

emulates a subset of {\tt /bin/test}
(a.k.a. {\tt /bin/[}) functionality    

\end{description}


The {\tt relation} function is described in section
\ref{Databases},  Databases,
page \pageref{Databases}.
Functions {\tt trace} and {\tt untrace} are described
in connection with debugging. 

See section \ref{Logging_and_Statistics}, Logging and Statistics,
page \pageref{Logging_and_Statistics}.
(This will probably change to Reference, Router, Debugging)

The {\tt hostname} function requires some further explanation.
It is intended to emulate the BSD UNIX {\tt /bin/hostname}
functionality, except that setting the hostname will only set
the {\em router's\/} idea of the hostname, not the system's.
Doing so will enable generation of `Message-Id' and `Received'
``trace'' headers on all messages processed by the {\em router\/}.
It is done this way since the {\em router\/} needs to know the official
domain name of the local host in order to properly generate these headers,
and this method is cleaner than reserving a magic variable for the purpose.
The {\em router\/} cannot assume the hostname reported by the system is
a properly qualified domain name, so the configuration file may generate
it using whichever method it chooses.
If the hostname indeed is a fully qualified domain name, then:

\begin{verbatim}
hostname "hostname"
\end{verbatim}


will enable generation of trace headers.

Finally, note that a symbol can have both a function-value and a
string-value.  The string value is of course accessed using the \$ prefix
convention of the Bourne shell language.

To test the configuration or routing data, proceed as follows:

\begin{verbatim}
MAILBIN/router -i               (select interactive mode)
trace on                        (turn tracing on)
router user@broken.address      (the address that gave you trouble)
router another@address          (and so on)
\end{verbatim}


Old salts can use {\tt /usr/lib/sendmail -bt} instead of {\tt router -i}. Once satisfied that routing works,

\begin{verbatim}
zmailer router
\end{verbatim}


will restart the {\em router\/}.

You can also run the {\em router\/} directly on a message.  Copy your message
to someplace other than the postoffice ({\tt /tmp} is usually good), in a
numeric file name.  If the file name is {\tt 123}, you run

\begin{verbatim}
MAILBIN/router 123
\end{verbatim}


this will create the file {\tt .123} containing the control information
produced by the {\em router\/}.






\subsubsection{Databases\label{Databases}}



Many of the decisions and actions taken by configuration file code depend
on the specifics of the environment the MTA finds itself in.  So, not just
the facts that the local host is attached to (say) the UUCP network and a
Local Area Network are important, but it is also essential to know the specific
hosts that are reachable by this method.  Hardcoding large amounts of such
information into the configuration file is not practical.  It is also
undesirable to change what is really a program (the configuration file),
when the information (the data) changes.

The desirable solution to this data abstraction problem is to provide a way
for the configuration file programmer to manage such information externally
to ZMailer, and access it from within the {\em router\/}.  The logical way to do
this is to have an interface to externally maintained databases.  These
databases need not be terribly complicated; after all the simplest kind of
information needed is that a string is a member of some collection.  This
could simply correspond to finding that string as a word in a list of
words.

However, there are many ways to organize databases, and the necessary
interfaces cannot be known in advance.  The {\em router\/} therefore implements a
framework that allows flexible interfacing to databases, and easy extension
to cover new types of databases.

To use a database, two things are needed: the name of the database, and a
way of retrieving the data associated with a particular key from that
database.  In addition to this knowledge, the needs of an MTA do include
some special processing pertinent to its activities and the kind of keys to
be looked up.

Specifically, the result of the data lookup can take different forms: one
may be interested only in the existence of a datum, not its value, or one
may be looking up paths in a pathalias database and need to substitute the
proper thing in place of `{\tt \%s}' in the string returned from the database
lookup.  It should be possible to specify that this kind of postprocessing
should be carried out in association with a specific data access.
Similarly, there may be a need for search routines that depend on the
semantics of keys or the retrieved data.  These possibilities have all been
taken into consideration in the definition of a relation.  A relation maps
a key to a value obtained by applying the appropriate lookup and search
routines, and perhaps a postprocessing step, applied to a specified
database that has a specified access method.

The various attributes that define a relation are largely independent.
There will of course be dependencies due to the contents or other semantics
of a database.  In addition to the features mentioned, each relation may
optionally have associated with it a subtype, which is a string value used
to tell the lookup routine which table of several in a database
one is interested in.

There are no predefined relations in the {\em router\/}.  They must all be
specified in the configuration file, before first use.  This is done by
calling the special function {\tt relation} with various options, as indicated
by the usage string printed by the {\tt relation} function when called the wrong
way:

\begin{verbatim}
     Usage: relation -t dbtype [-f file -e# -s# -blnu -d driver] name
\end{verbatim}


The `{\tt t}' option specifies one of several predefined database types, each
with their specific lookup routine.  It determines a template for the set
of attributes associated with a particular relation.  The predefined
database types are:



\begin{description}
\item[{\tt bhash}] \mbox{}

the database is in BSD DB HASH format.

\item[{\tt bind}] \mbox{}

the database is the BIND nameserver, accessed through the standard resolver routines.

\item[{\tt btree}] \mbox{}

the database is in BSD DB BTREE format.

\item[{\tt dbm}] \mbox{}

the database is in DBM format (strongly discouraged).

\item[{\tt gdbm}] \mbox{}

the database is in GNU GDBM format.

\item[{\tt headers}] \mbox{}

router internal database of various headers, and how they are
to be treated.

\item[{\tt hostsfile}] \mbox{}

{\tt /etc/hosts} lookup using {\tt gethostbyname()}.

\item[{\tt incore}] \mbox{}

the database is a high-speed bundle of data kept entirely in the router process core memory.  This is for a short-term data storage, like handling duplicate detection.

\item[{\tt ldap}] \mbox{}

Mechanism for X.500 Directory access lookup with the "light-weight
directory access protocol."

\item[{\tt ndbm}] \mbox{}

the database is in NDBM (new DBM) format.

\item[{\tt ordered}] \mbox{}

the database is a text file with key-datum pairs on each line, keys are looked up using a binary search in the sorted file.

\item[{\tt selfmatch}] \mbox{}

a special type that does translate the numerical address of format:
12.34.56.78
into binary form, and checks that it is (or is not) actually
our own local IP addresses. This is used in address literal
testing of addresses of type:
localpart@$12.34.56.78$

\item[{\tt unordered}] \mbox{}

the database is a text file with key-datum pairs on each line, keys are looked up using a sequential search.

\item[{\tt yp}] \mbox{}

Sun SunOS 4.x YP (these days "NIS") interface library.

\end{description}


A subtype is specified by appending it to the database type name separated
by a slash.  For example, specifying {\tt bind/mx} as the argument to the 
`{\tt t}' option will store `{\tt mx}' for reference by the access 
routines whenever a query to that relation is processed. The subtypes must 
therefore be recognized by either the database-specific access routines 
(for translation into some other form), or by the database interface itself.

For {\tt unordered} and {\tt ordered} database types, the datum 
corresponding to a particular key may be null.  This situation arises if 
the database is a simple list, with one key per line and nothing else.  
In this situation, the use of an appropriate post-processor option 
(e.g. `{\tt b}') is recommended to be able to detect whether or not 
the lookup succeeded.

The `{\tt f}' option specifies the name of the database.  This is typically a
path that either names the actual (and single) database file, or gives the
root path for a number of files comprising the database (e.g. {\tt foo} may
refer to the NDBM files `{\tt foo.pag}' and `{\tt foo.dir}').  For the 
{\tt hostsfile} type of database, the {\tt /etc/hosts} file is the one 
used (and since the normal hosts file access routines do not allow specifying 
a different file, this cannot be overridden). The use of the {\tt dbm} format 
is strongly discouraged, since a portable program can only have a single DBM 
database associated with it.

The `{\tt s}' option specifies the size of the cache.  If this value 
is non-zero (by default it is 10), then an LRU cache of this size is 
maintained for previous queries to this relation, including both positive 
and negative results.

The `{\tt e}' option specifies the cache data expiration time in seconds.

The `{\tt b}' option asks that a postprocessor is applied to the database 
lookup result, so the empty string is returned from the relation query if the
database search failed, and the key itself it returned if the search
succeeded.  In the latter case, any retrieved data is discarded.  The
option letter is short for Boolean.

The `{\tt n}' option asks that a postprocessor is applied to the database 
lookup result, so the key string is returned from the relation query if the
database search failed, and the retrieved datum string is returned if the
search succeeded.  The option letter is short for Non-Null.

The `{\tt l}' option asks that all keys are converted to lowercase before 
lookup in the database.  This is mutually exclusive with the `{\tt u}' 
option.

The `{\tt u}' option asks that all keys are converted to uppercase before 
lookup in the database.  This is mutually exclusive with the `{\tt l}' 
option.

The `{\tt d}' option specifies a search routine.  Currently the only legal
argument to this option is {\tt pathalias}, specifying a driver that searches
for the key using domain name lookup rules.

The final argument is not preceeded by an option letter.  It specifies the
name the relation is known under.  Note that it is quite possible for
different relations to use the same database.

Some sample relation definitions follow:

\begin{verbatim}
     if [ -f /etc/named.boot ]; then
         relation -nt bind/cname -s 100 canon  # T_CNAME canonicalize hostname
         relation -nt bind/uname uname         # T_UNAME UUCP name
         relation -bt bind/mx neighbour        # T_MX/T_WKS/T_A reachability
         relation -t bind/mp pathalias         # T_MP pathalias lookup
     else
         relation -nt hostsfile -s 100 canon   # canonicalize hostname
         relation -t unordered -f $MAILBIN/db/hosts.uucp uname
         relation -bt hostsfile neighbour
         relation -t unordered -f /dev/null pathalias
     fi
\end{verbatim}


The above fragment defines a set of relations that can be accessed in the
same way, using the same names, independent of their actual definition.

\begin{verbatim}
     # We maintain an aliases database in the following format. Note: the
     # 'aliases' db name is magic to the internal alias expansion routines.
     if [ -f $MAILBIN/db/aliases.dat ]; then
         relation -t ndbm -f $MAILBIN/db/aliases aliases
     else
         relation -t ordered -f $MAILBIN/db/aliases.idx aliases
     fi
\end{verbatim}


As the comment says, the relation name {\tt aliases} has special significance
to the {\em router\/}.  Although the relation is not special in any other way (i.e.
it can be used in the normal fashion), the semantics of the data retrieved
are bound by assumptions in the aliasing mechanism.  These assumptions are
that key strings are local-name's, and the corresponding datum gives a byte
offset into another file (the root name of the aliases file, with a {\tt .dat}
extention), which contains the actual addresses associated with that alias.
The reason for this indirection is that the number of addresses associated
with a particular alias can be very large, and this makes the traditional
simple database formats inadequate.  For example, quick lookup in a text
file is only practical if it is sorted and has a regular structure.  A
large number of addresses associated with an alias makes structuring a
problem.  The situation for DBM files and variations have problems too, due
to the intrinsic limits of the storage method.  The chosen indirection
scheme avoids such problems without loss of efficiency.

Finally, some miscellaneous definitions that illustrate various
possibilities:

\begin{verbatim}
     relation -t unordered -f /usr/lib/news/active -b newsgroup
     relation -t unordered -f /usr/lib/uucp/L.sys -b ldotsys
     relation -t ordered -f $MAILBIN/db/hosts.transport -d pathalias transport
\end{verbatim}


Here, the first two illustrate convenient coincidences of format, and the
last definition shows what might be used if outgoing channel information is
maintained in a pathalias-format database (e.g. {\tt bar smtp!bar} means to
send mail to {\tt bar} via the SMTP channel).






\paragraph{Using a Pathalias Database}





Accessing route databases is a rather essential capability for a mailer.
At the University of Toronto, all hosts access a centrally stored database
through a slightly modified nameserver program.  If such a setup is not
practical at your site, other methods are available.  The most widespread
kind of route database is produced by the {\tt pathalias} program.  It
generates key-value pairs of the forms:

\begin{verbatim}
     uunet                ai.toronto.edu!uunet!%s
     .css.gov             ai.toronto.edu!uunet!seismo!%s
\end{verbatim}


which when queried about {\tt uunet} and {\tt beno.css.gov} correspond to 
the routes:

\begin{verbatim}
     ai.toronto.edu!uunet
     ai.toronto.edu!uunet!seismo!beno.css.gov
\end{verbatim}

Notice that there are two basic forms of routes listed: routes to UUCP node
names and routes to subdomain gateways.  Depending on the type of route
query, the value returned from a pathalias database lookup needs to be
treated differently.  For now, this may be accomplished by a configuration
file relation definition and interface function as shown:

\begin{verbatim}
     relation -t ndbm -f $MAILBIN/uuDB -d pathalias padb
     
     # pathalias database lookup function
     padblookup (name, path) {             # path is a local variable
             path = ${$name:padb}
             case "$path" in
             ((.+)!)?([^!]+)!%s
                     if [ $3 == $name ]; then
                             path = $2!$3
                     else
                             path = $2!$3!$name
                     fi
                     ;;
             .*%s.*  echo illegal route in pathalias db: $path
                     ;;
             esac
             return $path
     }
\end{verbatim}

This is actually a simplistic algorithm, but it does illustrate the method.
The lookup algorithm used when the `{\tt -d}' flag is specified in the
relation definition command is rather simple; it doesn't test various case
combinations for the keys it tries.  Therefore, the keys in the pathalias
output data should probably be converted to a single case, and the `{\tt -l}'
or `{\tt -u}' flag given in the relation definition.






\subsubsection{Mailing Lists and \~{}/.forward \label{mailing_list_maintenance}}



Mailing lists are implemented as files in the {\tt \$MAILSHARE/lists/} 
directory (or symlinks in there to files residing elsewere, though from a system
reliability standpoint it is better to have them in that directory,
and let users have symlinks to those files -- consider the NFS
with the user home directories in other machines...)

An alternate mechanism is to implement lists in the traditional {\em sendmail\/}
manner, however it means feeding the message to the scheduler, and
external program ({\tt /usr/lib/sendmail}) before it comes back to the
{\em router\/}.

The list FILE must have protection 0664 or stricter, 0700 has invalid bits.
(ok, so the ``x"-bit is not used, but illegal it is, all the same.)
Preferrable protection is: 0600

The {\tt \$MAILSHARE/lists/} directory must be owned by root.
The directory containing the "aliases" file  ({\tt \$MAILSHARE/db/}) 
must be owned by root, and the aliases file must comply with above mentioned 
protections.

The owner of FILE gets FILE-owner, and FILE-request mails, {\bf unless any
of the limitations are breached}.

If FILE has protection 666 (for example), the ZMailer internal function
"{\tt \$(filepriv \$filepath)}" 
returns "{\tt \$nobody}" (userid of nobody), and  function
"{\tt \$(uid2login \$nobody)}" fails, thus losing -owner, and 
-request features.

Also lists with filepriv ``nobody'' cannot be archived.

A mailing list is set up by creating a file in the 
{\tt \$MAILVAR/lists/} directory.
The file name is the lists' name (LIST) in all lowercase (case-
insensitive matching is done by converting to lowercase before
comparison).

The file contains a list of mail addresses (typically one per line)
which are parsed to pull out the destination addresses.  This means
the users' full names can be given just as in any valid RFC822 address.

The local account which owns the file will by default receive messages
sent to LIST-owner and LIST-request.  This can be explicitly overridden
in the aliases file.  Mail to the list will go out with LIST-owner as
the sender, so list bounce messages will return to the LIST-owner
address.  Archives of the list can be created by adding a file name
address (a local pathname starting with "/") to the LIST file.  The
archive file is written with the ownership of the owner of the LIST
file.  Forwarding the mailing list into a newsgroup
can be done using a mail to news script (two generations are provided
in utils/distribute and utils/mail2news).




\paragraph{Security Considerations}



A LIST file must not be world writable, while most likely it can be
group-writable.  The {\tt MAILVAR/lists/} directory must also not be group
or world writable and must be owned by root or by the owner of the
LIST file.  Otherwise the file is declared insecure and all addresses
in the file get the least possible privilege associated (the ``nobody''
uid).  This can cause various things to break, for example mailing
list archival, or the -owner and -request features if ``nobody'' is
not a valid account.

There is a mechanism to override using the modes on a {\tt file/directory}
as an indicator of its safeness.  Turning on the sticky bit on a file
or directory tells the mailer to treat it as if it was only owner
writable independent of its actual modes.  This allows {\tt MAILVAR/lists/}
to be group or world writable and sticky-bitted if you want your general
user population to be able to create mailing lists.




\paragraph{About Large Lists, and Memory Consumption}



With old configuration scripts there used to be problems with list
expansion causing serious memory bloat.  {\em ZMailer-2.98-mea\/} did introduce
a working solution via the builtin  {\tt \$(listexpand ..)} function.

The rest of these notes apply to older config files using old style pipe
of {\tt \$(listaddresses  < file {\verbar} maprrouter ...)}  where the  
maprrouter caused the actual memory bloat...

Internal list expansion (through the  {\tt \$MAILVAR/lists/LISTNAME} 
mechanism) is a sure way to expand {\em router\/} process memory usage.

You can decrease the memory requirement dramatically, if you can
feed all the addresses in the envelope, or via {\tt utils/listexpand.c}
utility (alpha-test tool on 1-Sep-94) You don't need to worry about 
it unless your list is 100+ recipients, only then the memory usage starts 
to bloat seriously with the old-style in-core {\tt \$(listaddress ...)} 
expander.

Although more interesting and useful models exist, the mail forwarding
functionality of ZMailer has been designed to generally emulate the
interface and behaviour of {\em sendmail\/}.  The mechanisms that accomplish this
are likely to be generalized in a future version.

The file containing the actual aliasing data is automatically created by
the {\em router\/} when asked to reconstruct the aliases database.  It does this
based on a text file containing the alias definitions.  This text file,
which corresponds to the {\em sendmail\/} aliases file, consists of individual
alias definitions, possibly separated by blank lines or commentary.
Comments are introduced by a sharp sign (octothorp: `\#') at any point where
a token might start (for example the beginning of a line, but not in the
middle of an address), and extend to the end of the line.  Each alias
definition has the exact syntax of an RFC822 message header, containing an
address-list, except for comments.  The header field name is the local-part
being aliased to the address-list that is the header value.

The fact that an alias definition follows the syntax for an RFC822 message
header, introduces an incompatibility with {\em sendmail\/}.  The string
{\tt :include:} at the start of a local-part (a legacy of RFC733) has special
semantics. {\em Sendmail\/} would strip this prefix, and regard the rest of the
local-part as a path to a file containing a list of addresses to be
included in the alias expansion. Indeed, the {\em router\/} behaves in the same
manner, but because some of the characters in the prefix are RFC822
specials, the entire local-part must be quoted.  Thus, whereas {\em sendmail\/}
allowed:

\begin{verbatim}
     people: :include:/usr/lib/mail/lists/people
\end{verbatim}

the proper syntax with ZMailer is:
\begin{verbatim}
     people: ":include:/usr/lib/mail/lists/people"
\end{verbatim}


Like {\em sendmail\/}, if a local-part is not found in the aliases database, the
{\em router\/} also checks {\tt \~{}local-part/.forward} (if such exists) for 
any address expansion.  The {\tt .forward} file format is also an RFC822 
address list, similar to what {\em sendmail\/} expects.

As special cases, a local-part starting with a pipe character (`{\tt {\verbar}}') is
treated as mail destined for a program (the rest of the local-part is any
valid argument to a {\tt sh -c} command), and a local-part starting with a
slash character (`{\tt /}') is treated as mail destined for the file named 
by the local-part.

General format:

\begin{verbatim}
        local-address-token:    "replacement address" ,
                                "extension line w/ another addr"
        ^^^^^^^^^^^^^^^^^^^     ^^^^^^^^^^^^^^^^^^^^^^^^^^^^^^^^
          "The Key"                "The Data"
\end{verbatim}

Protection of the aliases database must be at least 0644.
Protection of the {\tt \$MAILVAR/db/} directory must be at least 03755.



The following processing is done for (replacement) local-parts
(local mail addresses):  Note that @'s are not allowed in any local-part.



If the local-part starts with ``{\verbar}" assume it is a command specification:
\begin{verbatim}
                prog-pipe: "|/path/to/program -args"
\end{verbatim}

If the local-part starts with ``/" assume it is a file pathname:
\begin{verbatim}
                file-path: "/path/to/file"
\end{verbatim}

If the local-part starts with {\tt :include:} the rest should be a file
pathname of a list of mail addresses.  They are substituted:
\begin{verbatim}
                included-list: ":include:/path/to/address/file"
\end{verbatim}

After this point, all matches are case-insensitive by means of translating
the value to be looked up to lower-case, and then conducting a case-sensitive
lookup. {\bf All keys in aliases et.al. must be in lower case}.
(The hash functions inside  {\tt ndbm/gdbm/db/dbm}  are case sensitive, and 
as such, there is no way to avoid this requirement.)




\paragraph{aliases.cf Logic}



\begin{itemize}
\item If an aliases database exists and local-part is found in it, the list of
addresses mapped to by the alias entry is substituted.
\item If an mboxmap file exists and a mapping for the local-part is found in it,
the mapping (a host!homedir!user value) determines the remote recipient
(user@host) or recipient mailbox ({\tt homedir/../PObox/user}) if host 
is local.
\item If local-part is a login name and a readable {\tt .forward} file 
exists in the home directory, the list of addresses it contains is substituted.
\item If local-part is a file basename in the {\tt \$MAILVAR/lists} 
directory, the   list of addresses contained in the file is substituted, and 
the sender address set to local-part-owner.
\item If local-part is of the form file-owner or file-request, where file is
an entry in the {\tt \$MAILVAR/lists} directory, the account name 
of the owner of the file is substituted.   (File-owner identity and correct 
protections are important.)
\item If the local-part is of format  ``user.name'',  it is actually mapped via
separate   fullnamemap, and NEVER via {\tt \$MAILVAR/db/aliases}.
\item If PUNTHOST is defined (in {\tt /etc/zmailer.conf}) the address
{\tt local-part@\$PUNTHOST} is substituted.  Note that in this case
the mboxmap mechanism should be used to ensure local spool
mailbox delivery for local users.
\end{itemize}





\subsection{Scheduler Configuration}





The major action of the Scheduler is to periodically start up Transport
Agents and tell them what to do.  This is controlled by a table in a
configuration file that is read by the Scheduler when it starts.  

Any line starting with a `{\tt \#}' character is assumed to be a comment 
line, and is ignored, as are empty lines.  All other lines must follow a rigid
format. 

The scheduler configuration file consists of a set  of
clauses. Each clause is selected by the pattern it starts
with. The patterns  for  the  clauses   are  matched,  in
sequence, with the channel/host string for each recipient
address. When a clause pattern matches  an  address,  the
parameters set in the clause will be applied to the scheduler's 
processing of that address. If the clause specifies a command,  
the clause pattern matching sequence is terminated. This is a clause:

\begin{verbatim}
    local/*   interval=10s
              expiry=3h
              # want 20 channel slots in case of blockage on one
              maxchannel=20
              # want 20 thread-ring slots
              maxring=20
              command="mailbox -8"
\end{verbatim}

A clause consists of:

\begin{itemize}
\item A selection pattern  (in  shell style) that  is
matched  against        the  channel/host  string  for an
address.
\item 0  or more  variable assignments or keywords
(described below).
\end{itemize}


If the selection pattern does not contain a '/', it is
assumed to be a channel pattern and the host  pattern  is
assumed to be the wildcard '*'.

The components of a clause are separated by whitespace.
The pattern introducing a clause must start in the first
column of a line, and the variable assignments or keywords
inside a clause must not start in the first  column of a
line. This means a clause may be written both compactly
all on one line, or spread out with an assignment or keyword per line.

If the clause is empty (i.e., consists only of a pattern),
then the contents of the next non-empty clause will be used.

The typical configuration file will contain the following clauses:
\begin{itemize}
\item a clause matching all addresses (using the pattern */*) that sets 
up default values.
\item a  clause  matching  the  local delivery channel
(usually local).
\item a clause matching the deferred    delivery  channel
(usually hold).
\item a  clause  matching  the error reporting channel
(usually error).
\item clauses specific to the other channels    known  by
the {\em router\/}, for example, smtp and uucp.
\end{itemize}


The  actual  names  of  these channels are completely 
controlled by the {\em router\/} configuration file.

Empty  lines,  and  lines   whose   first   non-whitespace
character is '\#', are ignored.

Variable values may be unquoted words or values or double 
quoted strings.  Intervals  (delta  time)  are  specified
using  a concatenation of: numbers suffixed with 's', 'm',
'h', or 'd' modifiers designating the number as a  second,
minute, hour, or day value.  For example: 1h5m20s.

The known variables and keywords, and their typical values
and semantics are:

{\tt interval (1m)}
specifies the primary retry interval, which determines how frequently a 
transport agent should be scheduled for an address. The value is a delta
time specification. This value, and the retries value mentioned below, 
are combined to determine the interval between each retry attempt.

{\tt idlemax}
When a transport agent runs out of jobs, they are moved
to ``idle pool'', and if a transport agent spends more than idlemax
time in there, it is terminated.

{\tt expiry (3d)}
specifies  the  maximum age of an address in the scheduler 
queue before  a repeatedly  deferred  address  is
bounced with an expiration error.  The actual report is
produced when all addresses have been processed.

{\tt retries (1 1 2 3 5 8 13 21 34)} 
specifies the retry interval policy  of the scheduler 
for an address. The value must be a sequence of positive integers, these 
being multiples of the primary interval  before  a  retry is scheduled. 
The scheduler starts by going through the sequence as an  address  is 
repeatedly deferred. When the end of the sequence is reached, the scheduler 
will jump into the sequence at a random  spot and continue towards the end.  
This allows various retry strategies to be specified easily:

brute force (or "jackhammer"): {\tt  retries=0 }

constant primary interval: {\tt  retries=1 }

instant backoff: {\tt  retries="1 50 50 50 50 50 50 50 50 50 50 50 50" }

slow increasing (fibonacci) sequence: {\tt  retries="1 1 2 3 5 8 13 21 34" }

s-curve sequence: {\tt  retries="1 1 2 3 5 10 20 25 28 29 30" }

exponential sequence: {\tt  retries="1 2 4 8 16 32 64 128 256" }

etc.

{\tt maxta (0)} 
if retrying an address would cause the number of simultaneously 
active transport agents to exceed the specified  value, the retry is  
postponed. The check is repeated  frequently so  the address may be retried as
soon as possible after the  scheduled  retry interval. If the value is 0, 
a value of 1000 is used instead. Keep in mind that all  running  
transport  agents  will  keep open two {\tt pipe(2)} file-handles, and thus 
system-wide  limits may force a lower maximum than 1000. On a system
with a maximum of 256 open files, this would most likely succeed at 120.

{\tt maxchannel (0)}
if retrying an address would cause the number of 
simultaneously  active  transport agents processing mail for
the same channel to exceed  the  specified  value,  the
retry  is  postponed.  The check is repeated frequently
so the address may be retried as soon as possible after
the  scheduled  retry  interval.   If the value is 0, a
value of 1000 is used instead.

{\tt maxring (0)}
Recipients are groupped into ``threads'',  and  similar
threads  are groupped into ``thread-rings'', where the same
transport agent can be switched over from one recipient
to another.  This defines how many transport agents can
be running at any time at the ring.

{\tt skew (5)}
is the maximum number of retries before the retry  time
is aligned to a standard boundary (seconds since epoch,
modulo primary interval).  The lower this number (1  is
lowest), the faster the alignment is done.  The purpose
of this alignment is to ensure that eventually a single
transport agent invokation will be able to process 
destination addresses that arrived randomly at the  scheduler.

{\tt user (root)}
is  the  user  id  of  a transport agent processing the
address.  The value is either numeric  (a  uid)  or  an
account name.

{\tt group (daemon)}
is  the  group  id  of a transport agent processing the
address.  The value is either  numeric  (a  gid)  or a
group name.

{\tt command (smtp -srl \$LOGDIR/smtp \$host)}
is  the command line used to start a transport agent to
process the address. The program pathname is specified
relative  to the  {\tt MAILBIN/ta}  directory.   The  string
"\$channel" is replaced by the current matched  channel,
and  "\$host" is  replaced by the current matched host,
from the destination address, and "\$LOGDIR" substitutes
ZENV variable LOGDIR value there.  It is strongly 
recommended that the \$host is not to be used on a  command
definition, as it limits the usability of the idled transporter.

{\tt bychannel}
is a keyword (with no associated value) that tells  the
scheduler  that  the transport  agent specified in the
command will only process  destination  addresses  that
match  the  first  destination  channel  it encounters.
This is automatically set when  the  string  "\$channel"
occurs  in the command, but may also be specified manu-
ally by this keyword.  This is rarely used.

{\tt queueonly}
a clause with queueonly flag does not auto-start at the
arrival  of  a  message,  instead it must be started by
means of {\tt smtpserver(8)} command ETRN thru an  SMTP  connection.

For example, this is a complete configuration file:

\begin{verbatim}
    # Default values
    */*       interval=1m expiry=3d retries="1 1 2 3 5 8 13 21 34"
              maxring=0 maxta=0 skew=5 user=root group=daemon
    # Boilerplate parameters for local delivery and service channels
    local/*   interval=10s expiry=3h maxchannel=2 command=mailbox
    error     interval=5m maxchannel=10 command=errormail
    hold/*    interval=5m maxchannel=1 command=hold

    # Miscellaneous channels supported by router configuration
    smtp/*.toronto.edu
    smtp/*.utoronto.ca maxchannel=10 maxring=2
              command="smtp -srl $LOGDIR/smtp"
    smtp      maxchannel=10 maxring=5
              command="smtp -esrl $LOGDIR/smtp"

    uucp/*    maxchannel=5 command="sm -c $channel uucp"
\end{verbatim}


The  first  clause  (*/*)  sets up default values for all
addresses.  There is no command specification,  so  clause
matching  will  continue  after address have picked up the
parameters set here.

The third clause (error) has an implicit host wildcard  of
'*',  so  it  would  match  the same as specifying error/*
would have.

The fifth clause ({\tt smtp/*.toronto.edu}) has no further  
components so it selects the components of the following nonempty 
clause (the sixth).

Both the fifth and sixth clauses are specific  to  address
destinations  within the TORONTO.EDU and UTORONTO.CA 
organization (the two  are  parallel  domains).   At  most  10
deliveries to the smtp channel may be concurrently active,
and at most 2 for all possible hosts  within  TORONTO.EDU.
If \$host  is  mentioned in the command specification, the
transport agent will only be told about the  message  control  
files that indicate SMTP delivery to a particular
host. The actual host is picked at random from the  current  
choices, to avoid systematic errors leading to a deadlock of any queue.
















\subsection{sm Configuration}



sm  is a ZMailer {\em sendmail\/} compatible transport agent which is usually only run
by the {\tt scheduler(8)}, to deliver messages  by  invoking a
program with  facilities  and  in a way compatible with a
{\em sendmail\/} mailer. The sm program must be run with the same
current directory as the scheduler, namely {\tt POSTOFFICE/transport}.

The program scans the message control files named on stdin
for addresses destined for the channel and/or the host
given on the command line. If any are found, all matching
addresses and messages are processed according to the
specifications for the mailer in the configuration file.

The exit status of a mailer should be one of the standard
values specified in {\tt {\(<\)}sysexits.h{\(>\)}};. Of these, 
EX\_OK indicates  successful  deliver,  and  EX\_DATAERR,   EX\_NOUSER,
EX\_NOHOST,  EX\_UNAVAILABLE,  and EX\_NOPERM indicate permanent failure.  
All other exit codes will be treated  as a
temporary failure and the delivery will be retried.

Usage 

{\tt sm [ -8HQV ] [ -f configfile ] -c channel -h host mailer}


{\bf Configuration}

The configuration file {\tt MAILSHARE/sm.conf } associates the 
mailer keyword  from
the command line with a specification of a delivery program.  
This is very similar to the way the definition of a
mailer in  {\em sendmail\/} requires flags, a program name, and a
command line specification.  These are in fact the  fields
of  the entries of the configuration file.  Lines starting
with whitespace or a `{\tt \#}' are ignored, and all  
others are assumed to follow this format:

\begin{verbatim}
       mailer    flags  program                   argument list
\end{verbatim}


For example:

\begin{verbatim}
   local     mS     sm/localm                 localm -r $g $u
   prog      -      /bin/sh                   sh -c $u
   tty       rs     /usr/local/to             to $u
   uucp      U      /usr/bin/uux              uux - -r -a$g -gC $h!rmail ($u)
   usenet    m      sm/usenet                 usenet $u
   ean       mn     /local/lib/ean/gwsmean    gwsmean -d $u
   test      n      sm/test                   test $u
\end{verbatim}


The mailer field extends from the beginning of the line to
the first whitespace.  It is used simply as a key index to
the configuration  file  contents.  Whitespace is used as
the field separator for all the fields.

The flags field contains  a  concatenation  of  one-letter
flags.  If no flags are desired, a `{\tt -}' character should be
used to indicate presence of the field. All normal  {\em sendmail\/}  flags  are recognized, but the ones that do not make
sense in the context of ZMailer will  produce  an  error.
The flags that change the behaviour of sm are:

\begin{description}
\item[ {\tt b}] \mbox{}

will activate BSMTP-type wrapping with a ``hidden-dot''
algorithm; e.g. quite ordinary SMTP stream, but in ``batch mode''.



\item[ {\tt B}] \mbox{}

The  first `{\tt B}'  turns on similar BSMTP wrapping as
`{\tt b}', but adds SIZE and, if the sm is  started  with
option `{\tt -8}', also 8BITMIME options.  The second `{\tt B}'
adds there also DSN (Delivery Status  Notification) parameters.



\item[ {\tt E}] \mbox{}

will prepend `{\(>\)}' to any message body line starting
with `{\tt From }'.



\item[ {\tt f}] \mbox{}

adds ``{\tt -f sender}'' arguments to the delivery program.



\item[ {\tt n}] \mbox{}

will not prepend a From-space line (normal mailbox
separator line) to the message.



\item[ {\tt r}] \mbox{}

adds ``{\tt -r sender}'' arguments to the delivery program.



\item[ {\tt S}] \mbox{}

will run the delivery program with the same real
and effective uid as the sm process.  If this  flag
is  not  set, the delivery program will be run with
the real uid of the sm process. This may be useful if sm is setuid.


\item[ {\tt m}] \mbox{}
informs sm that each instance of the delivery program  
can deliver to many destinations. This affects {\bf \$u} expansion 
in the argument list, see below.

\item[ {\tt P}] \mbox{}
prepends a Return-Path: header to the message.

\item[ {\tt U}] \mbox{}
will prepend a From-space line, with a ``remote from
myuucpname'' at the end, to the message. This is what is expected by remote 
{\tt rmail(1)} programs for incoming UUCP mail.

\item[ {\tt R}] \mbox{}
use CRLF sequence as end-of-line sequence. Without
it, will use LF-only end-of-line sequence.

\item[ {\tt X}] \mbox{}
does SMTP-like ``hidden-dot'' algorithm  of doubling
all dots that are at the start of the line.

\item[ {\tt 7}] \mbox{}
will strip (set to 0) the 8th bit of every character in the 
message.

\end{description}


The path field specifies the location of the delivery program. Relative 
pathnames are allowed and are relative to the {\tt MAILBIN/} directory.

The arguments field extends to the end of the line. It
contains  whitespace separated  argv  parameters which may
contain one of the following sequences:

\begin{description}
\item[ {\tt \$g}] \mbox{}
which is replaced by the sender address.

\item[ {\tt \$h}] \mbox{}
which is replaced by the destination host.

\item[ {\tt \$u}] \mbox{}
which is replaced by the recipient address.  If the
{\tt -m} mailer flag is set and there are several recipients for this 
message, the argument containing  the {\tt \$u} will be replicated 
as necessary for each recipient.

\end{description}
