%% \subsection{rmail}

{\em rmail\/} is a program to process incoming UUCP mail.
{\em rmail\/} is usually invoked by a remote UUCP neighbour host's
mailer using a command line like:

\begin{verbatim}
uux - -r -asender -gC thishost!rmail (recipient1) (recipient2) ...
\end{verbatim}


The end result is that the remote neighbour's {\em uuxqt(8)\/} runs
{\em rmail\/} on {\tt thishost} with this command line:

\begin{verbatim}
rmail recipient1 recipient2 ...
\end{verbatim}


In both cases, a UUCP format mail message is on the standard input.

The task of {\em rmail\/} is to transform the trace information in
the UUCP format message to the equivalent RFC822 trace
information, and to submit the message to the {\em zmailer(1)\/}
router with the appropriate envelope information.

The expected input format looks like:

\begin{verbatim}
From address3  date3 remote from host3
>From address2  date2 remote from host2
>From address1  date1 remote from host1
\end{verbatim}


followed by the rest of the message. This is considered
equivalent to the following (as it might appear in a mailbox):

\begin{verbatim}
From host3!host2!host1!address1 date
Received: by host3 ... ; date3
Received: by host2 ... ; date2
Received: by host1 ... ; date1
\end{verbatim}


In order for the mailer to process the incoming message
properly, {\em rmail\/} must be run by a {\tt userid} which the
{\em router(1)\/} will accept forged mail from. This is normally
the UUCP account id.

{\bf Usage}

{\tt rmail [-d] [-h somewhere] recipient\ldots }

{\bf Parameters}

\begin{description}
\item[ {\tt -d} ] \mbox{}

turns on debugging output.



\item[ {\tt -h somewhere} ] \mbox{}

will use the argument as the 
default remote UUCP host name to use if there is no {\tt remote from host}
tag in the first From-space line in  the message.
The  default value for this is usually {\tt somewhere} or
{\tt uunet} (since uunet was a frequent purveyor of this
protocol violation).

\end{description}

