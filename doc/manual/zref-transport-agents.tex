\subsection{Delivery agents}

The delivery agent programs normally form the final stage of message delivery.

These programs vary in function and facilities based on what they are
doing to the messages, and what kind of channels they handle.


\subsubsection{mailbox}

The {\em mailbox\/} is a ZMailer transport agent which is usually
only run by the {\em scheduler(8)\/} program to deliver mail to
local user mailbox files. The {\em mailbox\/} program must be
run with root privileges and invoked with the same current 
directory as the {\em scheduler\/}, namely 
{\tt \$POSTOFFICE/transport/}.

Recipient addresses are processed  as follows: 

\begin{itemize}
\item Strip doublequotes around the address, if any. 
\item Strip prefixing backslashes, if any.  
\item If the address starts with a `{\tt {\verbar}}', the rest of the recipient address 
string is interpreted as  a shell command to be run. 
\item If the address starts with a `{\tt /}', the recipient address is a  filename
to append the message to. 
\item Otherwise the recipient address must be a local user id. 
\item If user is not  found,  and the first character of the address is a capital 
letter, the entire address is folded to lowercase and the user lookup is  retried.
\end{itemize}


If delivering to a user mailbox ({\tt \$MAILBOX/userid/}) which
does not exist, {\em mailbox\/} will try to create it. If the
{\tt \$MAILBOX/} directory is mounted from a remote system this
will succeed if the directory is group writable.

Some sanity checks are done on deliveries to files and mailboxes:

\begin{itemize}
\item The file being delivered to must have one link only, and must be either 
{\tt /dev/null} or a regular file.
\item The file lock must be held. (See below for a chapter about locks.)  
\end{itemize}


There is a further sanity check on mailbox deliveries, namely if
the mailbox is not empty the {\em mailbox\/} program will
enforce 2 newlines as a separator before the message to be 
delivered. This guarantees that User Agents, like {\em Mail(1)\/},
can find the about-to-be delivered message even  if the current
contents of the mailbox is corrupt.

When delivering to a process (by starting a Bourne shell to 
execute a specified command line), the environment is set up to 
contain {\tt \$PATH}, {\tt \$SHELL}, {\tt \$HOME}, 
{\tt \$USER},
{\tt \$SENDER},  {\tt \$UID} environment variables. The 
{\tt \$HOME} and {\tt \$USER} values are the recipient user's home
directory and login id respectively. The {\tt \$SENDER} value is
the sender address for the message (as it would appear in a From\_
line), and the UID value is the owner id of the process. The
SIGINT and SIGHUP signals are ignored, but SIGTERM is treated
normally. If the process dumps core, it will be retried later.
Otherwise any non-zero exit status is taken as a permanent 
failure and will result in an error message back to the sender.
The actual data delivered to a file, mailbox,  or process, is
identical. It  consists of the concationation of a UUCP style
separator line, the message header specified in the message 
control file, and the message body from the original message file.
The separator line starts with ``From '' and is followed by the
sender address and a timestamp.

After all deliveries and just before exiting, the mailbox process
will poke comsat(8C) in case recipients have turned on biff(1).
The program may be compiled to look in the rwho files on the 
system for recipient names logged onto neighbouring hosts, in which
case the comsat on the remote host will be poked. Even if this
compile-time option is enabled, this will only be done for users
that have a {\tt .rbiff} file in their home directory. (Unless an
`{\tt -DRBIFF\_ALWAYS}' compile option is used.)

{\bf Usage}

\begin{tscreen}
\begin{verbatim}
mailbox [-8][-M][-c channel][-h localpart][-l logfile][-VabrH] 
\end{verbatim}
\end{tscreen}


{\bf Parameters}

\begin{description}
\item[{\tt -c channel}] \mbox{}

specifies which channel name should be keyed on. 
The default is local.

\item[{\tt -h "localpart"}] \mbox{}

specifies which of the possible multiple recipients is to 
be picked this time. Default is ``none'', which selects all local channel recipients, 
however when the routing is done with scripts storing some tokens (other than ``-'') 
into the ``host''-part, it is possible to process ``host-wise'', i.e. so that each 
user has his or her own lock-state, and not just everybody hang on the same lock(s)...

\item[{\tt -l logfile}] \mbox{}

specifies a logfile. Each entry is a line containing message 
id, pre-existing mailbox size in bytes, number of bytes appended, and the file name or 
command line delivered to.

\item[{\tt -V}] \mbox{}

prints a version message and exits.

\item[{\tt -a}] \mbox{}

the access time on mailbox files is, by default, preserved across
delivery, so that programs such as {\tt login(1)} can determine if new
mail has arrived. This option disables the above action.

\item[{\tt -b}] \mbox{}

disables biff notification.

\item[{\tt -r}] \mbox{}

disables remote biff notification (if supported).

\item[{\tt -8}] \mbox{}

enables the MIME-QP-decoder to decode incoming  MIME-email with
Quoted-Printable encoded characters.

\item[{\tt -M}] \mbox{}

enables the creation of MMDF-style mail-folder in the  incoming
mail folder. The default is ``classic'' UNIX-style folder.

\end{description}


{\bf Interface}

As with all transport agents, the program reads relative
pathnames of message control files from  standard input 
(terminated with two linefeeds), and produces diagnostic output on the
standard output. Normal diagnostic output is of the form:

\begin{tscreen}
\begin{verbatim}
id/offset<TAB>notify-data<TAB>status message
\end{verbatim}
\end{tscreen}


where id is the inode number of the message file, offset is a
byte offset within its control file where the address being
reported on is kept, status is one of ok, error, or deferred, and
the message is descriptive text associated with the report. The
text is terminated by a linefeed. Any other format (as might be
produced by subprocesses) is passed to standard output for 
logging in the scheduler log.
The exit status is a code from {\tt {\(<\)}sysexits.h{\(>\)}}.

{\bf Locks}

The locking scheme used on the system is configurable at the
runtime, and has separate parameters for mailboxes and files.
The data is configurable with zenv variable {\tt \$MBOXLOCKS} in which
the following characters have the meanings:

\begin{description}
\item[{\tt `:'}] \mbox{}

Separates mailbox locks, and file-locks at the string. The left
side has mailbox locks, and the right side has locks for other
regular files. (Files with explicit paths defined.)

\item[{\tt `.'}] \mbox{}

For mailboxes only: Does ``dotlock'' (userid.lock), or (on Sun
Solaris) the {\tt maillock()} mechanism.

\item[{\tt `F'}] \mbox{}

If the system has {\tt flock()} system call, uses it to lock 
the entire file. (Ignored on systems without {\tt flock()})

\item[{\tt `L'}] \mbox{}

If the system has {\tt lockf()} system call, uses it to lock 
the entire file. (Ignored on systems without {\tt lockf()}) 

\end{description}


Locks are acquired in the same order as the key characters are listed.

The default for {\tt lockf()} capable systems is: {\tt MBOXLOCKS=".L:L"}
You can choose insane combinations of lock mechanisms, which on
some systems cause locks to fail always, like on {\em Linux-2.0\/} series
where programs must not use both {\tt lockf()} and {\tt flock()} locks.
It is extremely important that selected locking methods are consistent
throughout the system with all programs trying to acquire locks on
mail spools.

{\bf Environment}

The default location for user mailbox files is 
currently {\tt /var/mail/}. This may be modified by setting the variable
{\tt \$MAILBOX} in {\tt /etc/zmailer.conf} to the directory containing 
user mailbox files, for example {\tt /usr/spool/mail/}. This is best done 
in the ZMailer Config file. The variable {\tt \$MBOXLOCKS} is used to define locking schemes used for
mailbox spool files, and separately for other regular files.

{\bf Security}

Like all parts of ZMailer, the mailbox chooses to
err on the overly cautious side.  In thecase of pipes being run under
the mailbox, the program in the pipe is started through {\tt /bin/sh} with
severely sanitized environment variables, and with only the file
descriptors STDIN, STDOUT, and STDERR. Programs are refused from
running, if address analysis has found suspicuous data; external
messages cannot directly run programs, nor those addresses that
have had a security breach detected during (Same applies also
with writing into explicitely named files.)
The pipe subprogram is run with user-id it gets thru the address
privilege analysis during message routing, and it gets the groupid 
through lookup of {\tt getpwuid(uid)}. That is, if you have multiple
usernames with same uid, there are no guarantees as to which of
them is used for the gid entry.

{\bf Subprogram Environment Variables}

The mailbox sets the following eight environment variables for the subprograms it runs 
in the pipes:
\begin{enumerate}
\item {\tt HOME} The home directory path is taken from abovementioned getpwuid()
lookup.
\item {\tt USER} Likewise the textual username.
\item {\tt SENDER} is the incoming ``MAIL FROM:{\(<\)}..{\(>\)}'' address without brackets. For
an incoming error message, value ``{\(<\)}{\(>\)}'' is used.
\item {\tt ORCPT} when present, is the XTEXT encoded ORCPT value received at the
message injection into this system. See RFC 1891 for details.
\item {\tt ENVID} when present, is the XTEXT encoded ENVID value received at the
message injection into this system. See RFC 1891 for details.
\item {\tt ZCONFIG} is the location of the ZMailer ZENV file.
\item {\tt MAILBIN} is the value from ZENV.
\item {\tt MAILSHARE} is the value from ZENV.
\end{enumerate}









\subsubsection{hold}



hold - zmailer deferred processing transport agent

{\bf Description}

hold is a ZMailer transport agent which is usually only
run by the {\tt scheduler(8)} program to test conditions for
reprocessing of previously deferred message addresses.
The hold program must be run with the same current directory 
as the scheduler, namely {\tt \$POSTOFFICE/transport/}.

The program will interpret the host part of an address
destined for its channel as a condition that must be met
before the original address (in the user part) can be
reprocessed by the {\em router\/}. The condition specification
contains a general condition class name followed by colon
followed by a parameter string. The currently supported
condition classes are:

\begin{description}
\item[ns] \mbox{}

succeeds when the nameserver lookup indicated by
the parameter does not produce a temporary nameserver 
error. The parameter is a domain name followed by a slash 
followed by a standard Internet
nameserver Resource Record type name.



\item[timeout] \mbox{}

succeeds when the time given by the parameter (in
normal seconds-since-epoch format) has passed.



\item[io] \mbox{}

succeeds 10\% of the time, to allow retry of temporary I/O failures.



\item[script] \mbox{}

runs the named shell script with the optional given
argument. The parameter is a simple name, the
shell script name within the {\tt \$MAILBIN/bin/} directory,
optionally followed by a slash followed by an argument to be 
passed to the shell script.

\end{description}


For example:

\begin{tscreen}
\begin{verbatim}
NS:nic.ddn.mil/cname
TIMEOUT:649901432
IO:error
SCRIPT:homedir/joe
\end{verbatim}
\end{tscreen}


The condition class name is case-insensitive but is 
capitalised by convention. The parameter strings are 
case-preserved for condition class-specific interpretation.
Whitespace is not permitted.

The envelope of the resubmitted message is created from
the sender and (no longer deferred) recipient addresses,
and a ``via suspension'' header.

{\bf Usage}

\begin{tscreen}
\begin{verbatim}
hold [ -c channel ] [ -V ]
\end{verbatim}
\end{tscreen}


{\bf Parameters}

{\tt -c channel} specifies which channel name should be keyed on. The default is hold.

{\tt -V} prints a version message and exits.

{\bf Interface}

As all transport agents, the program reads relative path-
names of message control files from standard input (terminated 
with two linefeeds), and produces diagnostic output
on the standard output. Normal diagnostic output is of
the form:

\begin{tscreen}
\begin{verbatim}
id/offset/status message
\end{verbatim}
\end{tscreen}


where id is the inode number of the message file, offset
is a byte offset within its control file where the address
being reported on is kept, status is one of ok, error, or
deferred, and the message is descriptive text associated
with the report. The text is terminated by a linefeed.
Any other format (as might be produced by subprocesses) is
passed to standard output for logging in the scheduler
log.

The exit status is a code from {\(<\)}sysexits.h{\(>\)}.








\subsubsection{smtp}



{\em smtp\/} - zmailer SMTP client transport agent

{\em smtp\/} is a ZMailer transport agent which is usually only
run by the {\tt scheduler(8)} to transfer messages to a remote
Internet host using the SMTP protocol. The {\em smtp\/} program
must be run with the same current directory as the scheduler, 
namely {\tt \$POSTOFFICE/transport/}.

The program scans the message control files named on stdin
for addresses destined for its channel and the host given
on the command line. If any are found, all matching
addresses and messages are transferred in a single SMTP
conversation. The destination host might in fact be
served by any available mail exchanger for that host.

{\bf Usage}

\begin{tscreen}
\begin{verbatim}
smtp [ -78deEHrPsVxW ] [ -c channel ] [ -h heloname ] [ -l logfile ] 
[ -p remote-port ] [ -T timeout ] [ -F forcedest] [ -L localidentity ] host
\end{verbatim}
\end{tscreen}


{\bf Parameters}

\begin{description}
\item[-7] \mbox{}

forces SMTP channel to be 7-bit, and thus forcing
all 8-bit texts to be MIME-QP-encoded for the transport.



\item[-8] \mbox{}

forces SMTP channel to be 8-bit-clean, and as such,
to decode the message while transporting it (is it
is MIME QP encoded).



\item[-c channel] \mbox{}

specifies which channel name should be keyed on.
The default is smtp.



\item[-d] \mbox{}

turns on debugging output.



\item[-e] \mbox{}

asks that for every destination address specification 
with a matching channel name, an MX lookup is
done on the hostname to see whether the currently
connected host can provide service for that destination. 
The default is to just do a textual name
comparison with the destination hostname as given
on the command line.



\item[-E] \mbox{}

use the ``EHLO''-greeting only if the remote server
initial banner reports ``ESMTP'' on it.



\item[-h host] \mbox{}

specifies the hostname for the SMTP HELO greeting.
The default is the hostname of the local system, as
returned by {\tt gethostname(2)} or {\tt uname(2)}.



\item[-F forcedest] \mbox{}

overrides delivery destination by forcing all
email to be sent to given forcedest IP-number/hostname.



\item[-H] \mbox{}

Disable the per default active forced 8-bit headers
conversion into MIME-2-format.



\item[-L localident] \mbox{}

specifies (for multi-homed machines) that they
should use specified identity when connecting to
the destination. Think of server with multiple IP
numbers due to virtual hosting, for example. At
such systems there may be situation when virtual
identity needs to be used for reaching the destination system.



\item[-l logfile] \mbox{}

specifies a log file where the complete SMTP command 
transaction will be copied. Each line in the
log will be prefixed with the process id of the
transport agent process, so the same log file can
be used by all SMTP clients.



\item[-r] \mbox{}

asks to set up SMTP connections using a source TCP
port number under 1024. This is in the range of
port numbers only available to a privileged process
on some UNIX systems, which has led to some 
misguided attempts at mail security based on this
mechanism.



\item[-s] \mbox{}

asks to report the progress of the SMTP conversation 
and data transfer on the command line in a way
that will be visible to {\tt ps(1)}.



\item[-x] \mbox{}

turns off MX lookups on delivery connections. This
may be used ignore public MX knowledge and do
exactly what the {\em router\/} says in cases where delivering 
to an explicit IP address is inappropriate.



\item[-P] \mbox{}

disable SMTP-PIPELINING usage (ESMTP keyword: PIPELINING)



\item[-T timeout] \mbox{}

specifies the timeout, in seconds, when waiting for
a response to an SMTP command. The timeout applies
to all SMTP command-response exchanges except for
the acknowledgement after terminating the DATA portion 
of a message transaction (after sending the
``.'' CRLF sequence). The default timeout is 10 
minutes, the minimum acceptable value is 5 seconds.
The timeout on the DATA acknowledgement is very
large, at least 24 hours.



\item[-V] \mbox{}

prints a version message and exits.



\item[-W] \mbox{}

turns on the DNS WKS checking, and if the remote
system does not have SMTP in its WKS-bits, email
delivery to such address is aborted with an error
message.

\end{description}


{\bf Interface}

As all transport agents, the program reads relative path
names of message control files from standard input 
(terminated with two linefeeds), and produces diagnostic output
on the standard output. Normal diagnostic output is of
the form:

\begin{tscreen}
\begin{verbatim}
id/offset<TAB>notify-data<TAB>status message
\end{verbatim}
\end{tscreen}


where id is the inode number of the message file, offset
is a byte offset within its control file where the address
being reported on is kept, status is one of ok, error, or
deferred, and the message is descriptive text associated
with the report. The text is terminated by a linefeed.
Any other format (as might be produced by subprocesses) is
passed to standard output for logging in the scheduler
log.

The exit status is a code from {\tt {\(<\)}sysexits.h{\(>\)}}.

{\bf Extended SMTP}

When a user sends out 8-bit mail with the proper headers, this
module can send it out to conforming servers either in
8-bit transparent manner, or down-converting Content 
Transfer-Encoding: 8BIT to Content-Transfer-Encoding: 7BIT
or Content-Transfer-Encoding: QUOTED-PRINTABLE depending
on what is the mail contents.
This works only with Content-Type: text/plain thus no
fancy multipart/alternate et.al. schemes..
When Content-Transfer-Encoding: -header is not present in
the headers, and recipient has not declared 8-bit SMTP
capability, mail contents are treated with old 7-bit
stripping method.








\subsubsection{{\em sm\/}}



{\em sm\/} - zmailer Sendmail compatible transport agent

{\em sm\/} is a ZMailer transport agent which is usually only run
by the {\tt scheduler(8)}, to deliver messages by invoking a
program with facilities and in a way compatible with a
{\em sendmail\/} mailer. The {\em sm\/} program must be run with the same
current directory as the scheduler, namely 
{\tt \$POSTOFFICE/transport/}.

The program scans the message control files named on stdin
for addresses destined for the channel and/or the host
given on the command line. If any are found, all matching
addresses and messages are processed according to the
specifications for the mailer in the configuration file.

The exit status of a mailer should be one of the standard
values specified in {\tt {\(<\)}sysexits.h{\(>\)}}. Of these, EX\_OK indicates 
successful delivery, and EX\_DATAERR, EX\_NOUSER,
EX\_NOHOST, EX\_UNAVAILABLE, and EX\_NOPERM indicate permanent 
failure. All other exit codes will be treated as a
temporary failure and the delivery will be retried.

{\bf Usage}

\begin{tscreen}
\begin{verbatim}
sm [ -8HQV ] [ -f configfile ] -c channel -h host mailer
\end{verbatim}
\end{tscreen}


{\bf Parameters}

\begin{description}
\item[{\tt -8}] \mbox{}

tells that the output is 8-bit clean, and for any
MIME message with QUOTED-PRINTABLE encoding the
coding can be decoded.



\item[{\tt -Q}] \mbox{}

tells that the transport channel will likely treat
poorly control characters like TAB, and possibly
SPACE too.. This encodes them all by using QUOTED-PRINTABLE encoding.



\item[{\tt -f configfile}] \mbox{}

specifies the name of a configuration file containing 
specifications of the various known {\em sendmail\/}
compatible mailer programs: how to invoke them and
how to process messages for them. The default is {\tt \$MAILSHARE/sm.cf}.



\item[{\tt -c channel}] \mbox{}

specifies which channel name should be keyed on.
There is no default. If this option is not specified, the {\tt -h} option must be.



\item[{\tt -h host}] \mbox{}

specifies which host name should be keyed on.
There is no default. If this option is not specified, the {\tt -c} option must be.



\item[{\tt v-V}] \mbox{}

prints a version message and exits.

\end{description}




{\bf Configuration}

The configuration file associates the mailer keyword from
the command line with a specification of a delivery program. 
This is very similar to the way the definition of a
mailer in Sendmail requires flags, a program name, and a
command line specification. These are in fact the fields
of the entries of the configuration file. Lines starting
with whitespace or a '\#' are ignored, and all others are
assumed to follow this format:

\begin{tscreen}
\begin{verbatim}
mailer flags program argument list
\end{verbatim}
\end{tscreen}


For example:

\begin{tscreen}
\begin{verbatim}
local mS sm/localm localm -r $g $u
prog - /bin/sh sh -c $u
tty rs /usr/local/to to $u
uucp U /usr/bin/uux uux - -r -a$g -gC $h!rmail ($u)
usenet m sm/usenet usenet $u
ean mn /local/lib/ean/gwsmean gwsmean -d $u
test n sm/test test $u
\end{verbatim}
\end{tscreen}


The mailer field extends from the beginning of the line to
the first whitespace. It is used simply as a key index to
the configuration file contents. Whitespace is used as
the field separator for all the fields.

The flags field contains a concatenation of one-letter
flags. If no flags are desired, a `-' character should be
used to indicate presence of the field. All normal {\em sendmail\/} flags 
are recognised, but the ones that do not make
sense in the context of ZMailer will produce an error.
The flags that change the behaviour of {\em sm\/} are:

\begin{description}
\item[b] \mbox{}

will activate BSMTP-type wrapping with ``hidden-dot''
algorithm; e.g. quite ordinary SMTP stream, but in
``batch mode''.



\item[B] \mbox{}

The first `{\tt B}' turns on similar BSMTP wrapping as
`{\tt b}', but adds SIZE and, if the {\em sm\/} is started with
option `{\tt -8}', also 8BITMIME options. The second `{\tt B}'
adds there also DSN (Delivery Status Notification)
parameters.



\item[E] \mbox{}

will prepend `{\(>\)}' to any message body line starting
with `From '.



\item[f] \mbox{}

adds ``{\tt -f sender}'' arguments to the delivery program.



\item[n] \mbox{}

will not prepend a From-space line (normal mailbox
separator line) to the message.



\item[r] \mbox{}

adds ``{\tt -r sender}'' arguments to the delivery program.



\item[S] \mbox{}

will run the delivery program with the same real
and effective uid as the {\em sm\/} process. If this flag
is not set, the delivery program will be run with
the real uid of the {\em sm\/} process. This may be useful
if {\em sm\/} is setuid.



\item[m] \mbox{}

informs {\em sm\/} that each instance of the delivery program 
can deliver to many destinations. This
affects {\bf {\tt \$u}} expansion in the argument list, see
below.



\item[P] \mbox{}

prepends a {\tt Return-Path:} header to the message.



\item[U] \mbox{}

will prepend a From-space line, with a ``remote from
myuucpname'' at the end, to the message. This is
what is expected by remote rmail(1) programs for
incoming UUCP mail.



\item[R] \mbox{}

use CRLF sequence as end-of-line sequence. Without it, will use 
LF-only end-of-line sequence.



\item[X] \mbox{}

does SMTP-like `hidden-dot' algorithm of doubling all dots that are 
at the start of the line.



\item[7] \mbox{}

will strip (set to 0) the 8th bit of every character in the message.

\end{description}


The path field specifies the location of the delivery program. 
Relative pathnames are allowed and are relative to
the {\tt \$MAILBIN/} directory.

The arguments field extends to the end of the line. It
contains whitespace-separated argv parameters which may
contain one of the following sequences:

{\tt \$g} which is replaced by the sender address.

{\tt \$h} which is replaced by the destination host.

{\tt \$u} which is replaced by the recipient address. If the

{\tt -m} mailer flag is set and there are several recipients for this message, the argument containing the {\tt \$u} will be replicated as necessary for each recipient.

{\bf Iterface}

As all transport agents, the program reads relative pathnames 
of message control files from standard input (terminated 
with two linefeeds), and produces diagnostic output
on the standard output. Normal diagnostic output is of
the form:

\begin{tscreen}
\begin{verbatim}
id/offset<TAB>notify-data<TAB>status message
\end{verbatim}
\end{tscreen}


where id is the inode number of the message file, offset
is a byte offset within its control file where the address
being reported on is kept, status is one of ok, error, or
deferred, and the message is descriptive text associated
with the report. The text is terminated by a linefeed.
Any other format (as might be produced by subprocesses) is
passed to standard output for logging in the scheduler
log.

The exit status is a code from {\tt {\(<\)}sysexits.h{\(>\)}}.






\subsubsection{libta - Transport Agent Support Library}



This is the library that all transport agents use, and several of its
functions are intended to aid message processing.




\paragraph{Function groupings}

Transport agent support library function groups are:

\begin{itemize}
\item Message file manipulation routines.
\item Diagnostics routines.
\end{itemize}





\paragraph{Function listings }


\paragraph{Function usage examples}


\subsubsection{Security Issues}

Text to be inserted here.

