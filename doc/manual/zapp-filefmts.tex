%% \subsection{Internal File Data Formats}

\subsection{Message File Format}

As mentioned, the {\em router\/} picks up message files from a specific directory.
Normally, message file names can be arbitrary valid file names, and indeed
this is convenient when debugging.  However, because the {\em router\/} daemon
scans its own current directory, miscellaneous output from the {\em router\/}
process may show up in this directory (e.g. profiling data, or core dumps
(unthinkable as that is)).  Furthermore, it is useful to be able to hide
files from the {\em router\/} scanning (indeed the {\em router\/} may wish to do so
itself).

When the {\em router\/} process is scanning for message files then, it only
considers at file names that have a certain format.  Specifically, the
message file name must start with a digit.  This method was chosen to
accomodate the message file names, as generated by the standard submission
interface library routines, which will be strings of digits representing
the message file's inode number.

A message file contains three sections: the message envelope, the message
header, and the message body (in that order).  The message body is
separated from the previous sections by a blank line.  The message body may
be empty, and either of the message envelope or message header may be
empty.  The restriction on the latter situation, is that one of those
sections must contain destination information for the message.

The message envelope and the message header have very similar syntax.  The
only difference is that while the message header must adhere to RFC822, the
message envelope header fields are terminated by whitespace (` ') instead
of a colon (`:').  The semantics of the two message file sections is quite
different, and will be covered later.




\subsubsection{Envelope Header Lines}



The message envelope headers are used to carry meta-information
about the message.  The goal is to carry transport-envelope 
information separate from message (RFC-822) headers, and body.
At first the message starts with a set of envelope headers 
(*-prefix denotes optional):

\begin{verbatim}
 *external \n
 *rcvdfrom %s@%s (%s) \n
 *bodytype %s \n
 *with %s \n
 *identinfo %s \n
\end{verbatim}

Either:
\begin{verbatim}
   from <%s> \n
\end{verbatim}

Or:
\begin{verbatim}
   channel error \n
 *envid %s \n
 *notaryret %s \n
\end{verbatim}


Then for each recipient pairs of:
\begin{verbatim}
 *todsn [NOTIFY=...] [ORCPT=...] \n
 to <%s> \n
\end{verbatim}

Just before the data starts, a magic entry:
\begin{verbatim}
 env-end \n
\end{verbatim}


Then starts the message RFC-822 headers, and below it, the body.

The header fields recognized by ZMailer in the message envelope are:

\begin{description}
\item[{\tt bodytype {\em word\/}} ] \mbox{}

not used. Compatibility with the {\em sendmail\/} feature 

\item[{\tt channel} {\em word\/}] \mbox{}

sets the channel corresponding to the message origin(*), usually as ``channel error''

\item[{\tt comment {\bf `string'}} ] \mbox{}

arbitrary comment

\item[{\tt env-end}] \mbox{}

separator between the envelope and the RFC822 headers

\item[{\tt env-eof}] \mbox{}

alias to env-end

\item[{\tt envid} xtext] \mbox{}

ESMTP DSN ENVID value

\item[{\tt external}] \mbox{}

keyword indicating the external origin of a message

\item[{\tt from} `address'] \mbox{}

a source address(*)

\item[{\tt fullname} `phrase'] \mbox{}

sets the full name of the local sender

\item[{\tt identinfo} `string'] \mbox{}

The SMTP server's ident lookup result, this does not guarantee anything about the sender though.

\item[{\tt loginname} `local-part'] \mbox{}

requests using this mail id for the local sender

\item[{\tt notaryret } {\em word\/}] \mbox{}

ESMTP DSN RET={\em word\/}, either ``FULL'' ``HPRS''

\item[{\tt rcvdfrom} `domain'] \mbox{}

sets the host the message was received from(*)

\item[{\tt to} `address-list'] \mbox{}

a destination address list

\item[{\tt todsn} `phrase'] \mbox{}

ESMTP DSN recipient parameters. Note: this must be before the recipient ``To'' line for which this gives the extra parameters.

\item[{\tt user} `local-part'] \mbox{}

sets the user the message was received from(*)

\item[{\tt verbose} zzz] \mbox{}

zzz

\item[{\tt via} `word'] \mbox{}

for RFC822 Received: header to be generated

\item[{\tt with} `word'] \mbox{}

for RFC822 Received: header to be generated

\end{description}


The (*)'s beside the descriptions indicate this is a privileged field.
That is, the action will only happen if ZMailer trusts the owner of the
message file (*Note Security: security.).  As with a normal RFC822 header,
other fields are allowed (though they will be ignored), and case is not
significant in the field name.  The {\em router\/} will do appropriate checks for
the fields that require it.

With this knowledge, we can now appreciate the minimal message file:

\begin{verbatim}
     --------------------
     to bond
     
     --------------------
\end{verbatim}


This will cause an empty message to be sent to {\tt bond}.  A slightly more
sophisticated version is:

\begin{verbatim}
     --------------------
     from m
     to bond
     via courier
     env-end
     From: M
     To: Bond
     Subject: do get a receipt, 007!
     
     You are working for the Government, remember?
     --------------------
\end{verbatim}


Notice that there is no delimiter between the message envelope and the
message header.  A more sophisticated example in the same vein:

\begin{verbatim}
     --------------------
     from ps/d-ops
     to <007@sis.mod.uk>
     env-end
     From: M <d-ops@sis.mod.uk>
     Sender: Moneypenny <ps/d-ops@sis.mod.uk>
     To: James Bond <007@sis.mod.uk>
     Subject: where are you???!
     Classification: Top Secret
     Priority: Flash
     
     We have another madman on the loose.  Contact "Q" for usual routine.
     --------------------
\end{verbatim}


If the {\tt Classification} header is paid attention to in ZMailer, this
requires that the {\em router\/} recognize it in the message header, and take
appropriate action.  In general the {\em router\/} can extract most of the
information in the message header, and make use of it if the information is
lacking in the envelope.  The envelope headers in the above message are
superfluous, since the same information is contained in the message header.
Using the following envelope headers would be exactly equivalent to using
the ones shown above (assuming the local host is {\tt sis.mod.uk}):

\begin{verbatim}
     --------------------
     From Moneypenny <ps/d-ops@sis.mod.uk>
     To James Bond <007@sis.mod.uk>
     ...
     --------------------
\end{verbatim}


ZMailer will extract the appropriate address information from whatever the
field values are, as long as they obey the defined syntax (indicated in the
list of recognized envelope fields above).  ZMailer will complain in case
of unexpected errors in the envelope headers.

The message body is not interpreted by ZMailer itself.  As far as the
{\em router\/} is concerned, it can be arbitrary data.  However, certain Transport
Agents may require limitations on the message body data. For example, the
SMTP only deals with ASCII data with a small guaranteed line length.










\subsection{Message Control File}

A message control file is a file created by the {\em router\/} to contain all the
information necessary for delivery of a message submitted in a
corresponding message file.  It is interpreted by the Scheduler, which
needs to know at all times which messages are pending to go where, and how.
It is also interpreted by one or more Transport Agents, possibly
concurrently, that extract the delivery information relevant to their
purpose.

The concurrency aspect means that the Transport Agents must cooperate on a
locking protocol to ensure that delivery to a particular destination is
attempted by only one Transport Agent at a time, and a status protocol to
ensure unique success or failure of delivery for each destination.  There
are potentially many ways to implement such protocols, but, in the spirit
of simplicity, ZMailer uses a control file as a form of shared memory.
Specific locations within each control file are reserved for flags that
indicate a specific state for their associated destination address.  The
rest is taken care of by the I/O semantics when multiple processes update
the same file.

Apart from necessary envelope and control information, a control file also
contains the new message header for the message, which contains the header
addresses as rewritten by the {\em router\/}.  Since a message may have several
destinations with incompatible address format requirements, there may be
several corresponding groups of message headers.  This will be illustrated
by the sample control file shown in the following subsection.




\subsubsection{Format}

A control file consists of a sequence of fields.  Each field starts at the
beginning of a line (i.e. at byte 0 or after a Newline), and is identified
by the appearance of a specific character in that location.  This id
character is normally followed by a byte containing a tag value (semaphore
flag), followed by the field value.

Here is a simple control file produced by a test message, just before it
was removed by the Scheduler:

\begin{verbatim}
     --------------------
     i 24700
     o 72
     l <88Jan10.003129est.24700@bay.csri.toronto.edu>
     e Rayan Zachariassen <rayan>
     s local - rayan
     r+local - rayan 2003
     m
     Received: by bay.csri.toronto.edu id 24700; Sun, 10 Jan 88 00:31:29 EST
     From:   Rayan Zachariassen <rayan>
     To:     rayan, rayan@ephemeral
     Subject: a test
     Message-Id: <88Jan10.003129est.24700@bay.csri.toronto.edu>
     Date:   Sun, 10 Jan 88 00:31:24 EST
     
     s local - rayan@bay.csri.toronto.edu
     r+smtp ephemeral.ai.toronto.edu rayan@ephemeral.ai.toronto.edu 2003
     m
     Received: by bay.csri.toronto.edu id 24700; Sun, 10 Jan 88 00:31:29 EST
     From:   Rayan Zachariassen <rayan@csri.toronto.edu>
     To:     rayan@csri.toronto.edu, rayan@ephemeral.ai.toronto.edu
     Subject: a test
     Message-Id: <88Jan10.003129est.24700@bay.csri.toronto.edu>
     Date:   Sun, 10 Jan 88 00:31:24 EST
     
     --------------------
\end{verbatim}


The id character values are defined in the {\tt mail.h} system header file,
which currently contains:

\begin{verbatim}
     
#define _CF_MESSAGEID   'i'     /* inode number of file containing message */
#define _CF_BODYOFFSET  'o'     /* byte offset into message file of body */
#define _CF_BODYFILE    'b'     /* alternate message file for new body */
#define _CF_SENDER      's'     /* sender triple (channel, host, user) */
#define _CF_RECIPIENT   'r'     /* recipient n-tuple, n >= 3 */
#define _CF_DSNRETMODE  'R'     /* DSN message body return control */
#define _CF_XORECIPIENT 'X'     /* one of XOR set of recipient n-tuples */
#define _CF_RCPTNOTARY  'N'     /* DSN parameters for previous recipient */
#define _CF_DSNENVID    'n'     /* DSN 'MAIL FROM<> ENVID=XXXX' data */
#define _CF_ERRORADDR   'e'     /* return address for error messages */
#define _CF_DIAGNOSTIC  'd'     /* diagnostic message for ctlfile offset */
#define _CF_MSGHEADERS  'm'     /* message header for preceeding recipients */
#define _CF_LOGIDENT    'l'     /* identification string for log entries */
#define _CF_OBSOLETES   'x'     /* message id of message obsoleted by this */
#define _CF_VERBOSE     'v'     /* log file name for verbose log (mail -v) */
#define _CF_TURNME      'T'     /* trigger scheduler to attempt delivery now */
#define _CF_RCVFROM     'F'     /* Where-from we are coming ? */               
\end{verbatim}


There is one field per line, except for {\tt \_CF\_MSGHEADERS} which  has
some special semantics described below.  The following describes the fields
in detail:

{\tt i}     
This field identifies the message file corresponding to this control
file.  It is the name of the message file in the {\tt QUEUE}
directory ({\tt \~{}/queue}).  This is typically the same as the
inode number for that file, but need not be.  It is used by Transport
Agents when copying the message body, and by the Scheduler when
unlinking the file after all the destination addresses have been
processed.  For example:

\begin{verbatim}
          i 21456
\end{verbatim}


{\tt o}     
Specifies the byte offset of the message body in the message file.  It
is used by Transport Agents in order to copy the message body quickly,
without parsing the message file.  For example:

\begin{verbatim}
          o 466
\end{verbatim}


{\tt b}     
Alternate message file for new body.

{\tt e}     
Gives an address to which delivery errors should be sent.  The address
must be an RFC822 mailbox.  For example:

\begin{verbatim}
          e "Operations Directorate" <d-ops@sis.mod.uk>
\end{verbatim}


{\tt l}     
The field value is an uninterpreted string which should prefix all log
messages and accounting records associated with this message.  This
value is typically the message id string.  For example:

\begin{verbatim}
          l <88Jan6.103158gmt.24694@sis.mod.uk>
\end{verbatim}


{\tt s}     
This field specifies an originator (sender) address triple, in the
sequence: previous channel, previous host, return address.  It remains
the current sender address until the next instance of this field.
Since there can only be one sender of a message, multiple instances of
the field will correspond to different return address formats as
produced by the {\tt crossbar} algorithm in the {\em router\/}.
For example:

\begin{verbatim}
          s smtp sis.mod.uk @lab.sis.mod.uk:q@deadly-sun.lab.sis.mod.uk
          s uucp sisops lab.sis.mod.uk!deadly-sun.lab.sis.mod.uk!q
\end{verbatim}


{\tt r}     
This field specifies a destination (recipient) address triple, in the
sequence: next channel, next host, address for next host.  Optional
information to be passed to the Transport Agent may be placed after the
mandatory fields; this currently refers to the delivery privilege of the
destination address.  Since the optional values of this field are only
interpreted by the Transport Agent, changes in what the {\em router\/} writes
must be coordinated with the code of the Transport Agents that might
interpret this field.  For example:

\begin{verbatim}
          r local - bond 0
          r uucp uunet sisops!bond -2
\end{verbatim}


{\tt R}
DSN message body return control.

{\tt X}
One of XOR set of recipient n-tuples.

{\tt N}
DSN parameters for previous recipient.

{\tt n}
DSN {\tt MAIL FROM{\(<\)}{\(>\)} ENVID=XXXX} data.

{\tt m}     
Apart from a message body, a Transport Agent needs the message headers
to construct the message it delivers.  These message headers are
stored as the value of this field.  Since message headers obviously
can span lines, the syntax for this field is somewhat different than
for the others.  The field id is immediately followed by a newline,
which is followed by a complete set of message headers.  These are
terminated (in the usual fashion) by an empty line, which also
terminates this field.  In the following example, the last line of
text is followed by an empty line, after which another field may
start:

\begin{verbatim}
          m
          From: M
          To: Bond
          Subject: do get a receipt, 007!
\end{verbatim}


{\tt d}     
This field is {\bf not} written by the {\em router\/}.  It is written by the
Scheduler or transport agents to remember errors associated with specific addresses. The
field value has two parts, the first being the byte offset in the
control file of the destination (recipient) address causing the error,
and the rest of the line being an error message.  The Transport Agents
discover these errors and report them to the Scheduler.  The Scheduler
will collect them and report them to the error return address (if any)
after all the destinations have been processed.
For example:
\begin{verbatim}
          d 878 No such local user: 'bond'.
\end{verbatim}
     

{\tt x}
Message id of message obsoleted by this.

{\tt v}
Log file name for verbose log ({\tt mail -v}).

{\tt T}
Trigger scheduler to attempt delivery now.

{\tt F}
Where-from we are coming ?               

It should be noted, that in sender and recipient fields the first two field
values (channel and host) cannot contain embedded spaces, but the third
field value (the address) may.  Therefore, in the presence of extra fields,
parsing within Transport Agents must be cautious and not assume that an
address does not contain spaces.

As mentioned, the second byte of most fields are used for concurrency
control and status indication.  This tag byte can contain several values
that indicate current or previous activity.  The fields where this is
relevant are the destination (recipient) address and diagnostic fields.
The tag values are defined in the `mail.h' file mentioned previously, as
follows:
\begin{verbatim}
     #define _CFTAG_NORMAL ' ' /* what the router sets it to be */
     #define _CFTAG_LOCK   '~' /* that line is being processed, lock it */
     #define _CFTAG_OK     '+' /* positive outcome of processing */
     #define _CFTAG_NOTOK  '-' /* something went wrong */
     #define _CFTAG_DEFER  _CFTAG_NORMAL /* try again later */
\end{verbatim}

The extract above is self-explanatory.

A message control file will normally contain a preamble that specifies
information about the associated message file, the message body offset, an
error return address, and a log entry tag.  After this comes a repeated
sequence of: sender address field, recipient address fields, and the
message header corresponding to these recipients.  After as many of these
groups as are necessary, any diagnostic fields will be appended to the end
of the control file.  The restrictions on the sequence of addresses and
message headers, are that a sender address field must precede any recipient
address field, and a recipient address field must (immediately) precede any
message header field, and no sender or recipient addresses may follow the
last message header field.






\subsection{Scheduler and Transport Agent Protocol}



The transport agent interface follows a master-slave model,
where  the  TA  informs the scheduler that it is ready for
the work, and then the scheduler sends it one job 
description,  and  waits  for diagnistics.  Once the job is 
finished, the TA notifies the scheduler that it is ready for
a new job.

A short sample session looks like this:
\begin{verbatim}
(start the transport agent)
#hungry --> (TA to scheduler)
spoolid\thostspec <-- (scheduler to TA)
diagnostics --> (TA to scheduler)
#hungry <-- (scheduler to TA)
...
\end{verbatim}

Normal diagnostic output is of the form:
\begin{verbatim}
id/offset\tnotarydata\tstatus message
\end{verbatim}

where
\begin{description}
\item[{\tt id}] \mbox{}

is the inode number of the message file,

\item[{\tt offset}] \mbox{}

is a byte offset within its control file where the address
being  reported on  is kept,

\item[{\tt notarydata}] \mbox{}

is a {\tt Ctrl-A} separated triplet is delivery-status-notification
information for  the message, 

\item[{\tt status}] \mbox{}

is one of {\tt ok}, {\tt error}, {\tt deferred}, or {\tt retryat},

\item[{\tt message}] \mbox{}

is  descriptive  text  associated
with  the  report.   The text is terminated by a linefeed.

\end{description}


Any other format (as might be produced by subprocesses) is
passed  to  standard  output  for logging in the scheduler
log. The {\tt retryat} response will assume the first word  of
the text is a numeric parameter, either an incremental
time in seconds if prefixed by `+', or otherwise an absolute
time in seconds since {\tt epoch}.

The exit status is a code from {\tt {\(<\)}sysexits.h{\(>\)}}.



\subsection{Database File Formats}



\subsubsection{Aliases File}



\begin{verbatim}
postmaster: root
postoffice: root
MAILER-DAEMON: root
mailer:     postmaster
postmast:   postmaster

proto:  postmaster
sync:   postmaster
sys:    postmaster
daemon: postmaster
bin:    postmaster
uucp:   postmaster
ingress: postmaster
audit:  postmaster

autoanswer: "|@MAILBIN@/autoanswer.pl"

nobody: /dev/null
no-one: /dev/null
junk-trap: /dev/null

#test-gw: "|/..."
#test.gw: "|/..."
\end{verbatim}

Doing expansion lists in {\em sendmail\/} style is not suggested, although
we certainly can do it.  There is a better mechanism in the ZMailer
to handle simple feats like these that {\em sendmail\/} systems do by placing
the file containing recipient addresses into the directory  
{\tt \$MAILVAR/lists/}.
This directory must have protection of 2775 or stricter, and the listfile
must have protection of 664 or stricter for *-request/owner-*/*-owner
auto-aliases to work. -- but to {\em sendmail\/} style lists:

\begin{verbatim}
listname: "/usr/lib/sendmail -fowner-listname listname-dist"
owner-listname: root # Well, what would you suggest for a sample ?
listname-owner: owner-listname
listname-request: root
listname-dist: ":include:/dev/null"
\end{verbatim}
