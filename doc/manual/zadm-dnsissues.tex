%%%%%%%%%%%%%%%%%%%%%%%%%%%%%%%%%%%%%%%%%%%%%%%%%%%%%%%%%%%%%%%%

% \section{DNS and ZMailer}

The cornerstone of everything in busy Internet email routing is a properly
working DNS server, and a modern resolver library. If you use the BIND 
nameserver, you should be using a recent version, at least BIND 4.8.
(As of this writing, bind is on version 8.1.1)

You can get improved DNS performance by installing local {\tt named},
which does cache replies, including {\bf negative} replies.

For the file {\tt /etc/resolv.conf}:
\begin{alltt}\medskip\hrule\medskip
domain     your.domain
nameserver 127.0.0.1
nameserver (some other server)
\medskip\hrule\medskip\end{alltt}


For the local nameserver daemon ({\em named}) you should have
at least following type of configuration:
\begin{alltt}\medskip\hrule\medskip
forwarders 10.12.34.56  10.45.67.89
options forward-only
\medskip\hrule\medskip\end{alltt}

which means that all the queries are attempted to be resolved
by the servers at IP addresses {\em 10.12.34.56} and
{\em 10.45.67.89}, and both the local server, and remote
servers will cache DNS responses.
