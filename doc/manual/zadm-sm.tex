% \section{sm Configuration}

{\em sm} is a ZMailer {\em sendmail} compatible transport agent
which is usually only run by the {\tt scheduler(8)}, to deliver
messages by invoking a program with  facilities  and  in a way 
compatible with a {\em sendmail} mailer.
The {\em sm} program must be run with the same current directory
as the scheduler, namely {\tt \$POSTOFFICE/transport/}.

The program scans the message control files named on stdin
for addresses destined for the channel and/or the host
given on the command line. If any are found, all matching
addresses and messages are processed according to the
specifications for the mailer in the configuration file.

The exit status of a mailer should be one of the standard
values specified in {\tt {\(<\)}sysexits.h{\(>\)}};.
Of these, EX\_OK indicates successful deliver, and EX\_DATAERR,
EX\_NOUSER, EX\_NOHOST, EX\_UNAVAILABLE, and EX\_NOPERM indicate
permanent failure.
All other exit codes will be treated as a temporary failure
and the delivery will be retried.

Usage 

{\tt sm [ -8HQV ] [ -f configfile ] -c channel -h host mailer}


{\bf Configuration}

The configuration file {\tt MAILSHARE/sm.conf } associates the 
mailer keyword  from
the command line with a specification of a delivery program.  
This is very similar to the way the definition of a
mailer in  {\em sendmail} requires flags, a program name, and a
command line specification.  These are in fact the  fields
of  the entries of the configuration file.  Lines starting
with whitespace or a `{\tt \#}' are ignored, and all  
others are assumed to follow this format:

\begin{alltt}
mailer flags program        argument list
\end{alltt}

For example:

\begin{alltt}
local  mS    sm/localm      localm -r $g $u
prog   -     /bin/sh        sh -c $u
tty    rs    /usr/local/to  to $u
uucp   U     /usr/bin/uux   uux - -r -a$g -gC $h!rmail ($u)
usenet m     sm/usenet      usenet $u
test   n     sm/test        test $u
\end{alltt}


The mailer field extends from the beginning of the line to
the first whitespace.  It is used simply as a key index to
the configuration  file  contents.  Whitespace is used as
the field separator for all the fields.

The flags field contains  a  concatenation  of  one-letter
flags.  If no flags are desired, a ``{\tt -}'' character should be
used to indicate presence of the field.
All normal {\em sendmail} flags are recognized, but the ones that
do not make sense in the context of ZMailer will produce an error.
The flags that change the behaviour of {\em sm} are:

\begin{description}
\item[ {\tt b}] \mbox{}

will activate BSMTP-type wrapping with a ``hidden-dot''
algorithm; e.g. quite ordinary SMTP stream, but in ``batch mode''.



\item[ {\tt B}] \mbox{}

The  first `{\tt B}'  turns on similar BSMTP wrapping as
`{\tt b}', but adds SIZE and, if the sm is  started  with
option `{\tt -8}', also 8BITMIME options.  The second `{\tt B}'
adds there also DSN (Delivery Status  Notification) parameters.



\item[ {\tt E}] \mbox{}

will prepend `{\(>\)}' to any message body line starting
with `{\tt From }'.



\item[ {\tt f}] \mbox{}

adds ``{\tt -f sender}'' arguments to the delivery program.



\item[ {\tt n}] \mbox{}

will not prepend a From-space line (normal mailbox
separator line) to the message.



\item[ {\tt r}] \mbox{}

adds ``{\tt -r sender}'' arguments to the delivery program.



\item[ {\tt S}] \mbox{}

will run the delivery program with the same real
and effective uid as the sm process.  If this  flag
is  not  set, the delivery program will be run with
the real uid of the sm process. This may be useful if sm is setuid.


\item[ {\tt m}] \mbox{}
informs sm that each instance of the delivery program  
can deliver to many destinations. This affects {\bf \$u} expansion 
in the argument list, see below.

\item[ {\tt P}] \mbox{}
prepends a Return-Path: header to the message.

\item[ {\tt U}] \mbox{}
will prepend a From-space line, with a ``remote from
myuucpname'' at the end, to the message. This is what is expected by remote 
{\tt rmail(1)} programs for incoming UUCP mail.

\item[ {\tt R}] \mbox{}
use CRLF sequence as end-of-line sequence. Without
it, will use LF-only end-of-line sequence.

\item[ {\tt X}] \mbox{}
does SMTP-like ``hidden-dot'' algorithm  of doubling
all dots that are at the start of the line.

\item[ {\tt 7}] \mbox{}
will strip (set to 0) the 8th bit of every character in the 
message.

\end{description}


The path field specifies the location of the delivery program. Relative 
pathnames are allowed and are relative to the {\tt MAILBIN/} directory.

The arguments field extends to the end of the line. It
contains  whitespace separated  argv  parameters which may
contain one of the following sequences:

\begin{description}
\item[ {\tt \$g}] \mbox{}
which is replaced by the sender address.

\item[ {\tt \$h}] \mbox{}
which is replaced by the destination host.

\item[ {\tt \$u}] \mbox{}
which is replaced by the recipient address.  If the
{\tt -m} mailer flag is set and there are several recipients for this 
message, the argument containing  the {\tt \$u} will be replicated 
as necessary for each recipient.

\end{description}
