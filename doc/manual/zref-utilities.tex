%% \section{Utilities}

There is considerable collection of various utilities in the
ZMailer package. Not all of them even become installed into
your system in all situations.

%\begin{multicols}{2}


\subsection{vacation}

{\em vacation\/} automatically replies to incoming mail. The canned 
reply is contained in the file {\tt .vacation.msg}, that you should 
create in your home directory (or the file {\tt Msgfile} specified by 
the {\tt -m} option).

This file should include a header with at least a {\tt Subject:} line 
(it should not include a {\tt To:} line -- if you want, you may include 
a {\tt From:} line, especially if you use the {\tt -m} option). 
For example:

{\bf Usage}

To start {\em vacation\/}, run the command {\tt vacation start}. 
It will create a {\tt .vacation.msg} file (if you don't already 
have one) in your home directory containing the message you want to send 
people who send you mail, and a {\tt .forward} file in your home 
directory containing a line of the form:

\begin{verbatim}
"\name", "|/opt/mail/bin/vacation name"
\end{verbatim}


where name is your login name. Make sure these files and your home 
directory are readable by everyone. Also  make sure  that  no  one  
else  can  write  to  them,  and  that  no  one  can  write  to  your  
home  directory. ({\tt chmod og-w \$HOME \$HOME/.forward})

To stop vacation, run the command {\tt vacation stop} It will move the 
{\tt .forward} file to {\tt .vacforward}, and the automatic 
replies will stop.

{\tt vacation start vacation stop vacation -I vacation [ -tN ] [ -mMsgfile ] [ -d ] [user]}

{\bf Parameters}

\begin{description}
\item[{\tt -I}, {\tt -i}] \mbox{}

initialize the {\tt .vacation.pag} 
and {\tt .vacation.dir} files and start vacation.

If the {\tt -I} (or {\tt -i}) flag is not specified, vacation 
tries to reply to the sender.



\item[{\tt -tN}] \mbox{}

Change the interval between repeat replies to the same 
sender. The default is one week. A trailing {\tt s}, {\tt m}, 
{\tt h}, {\tt d}, or {\tt w} scales N to seconds, minutes, 
hours, days, or weeks respectively.



\item[{\tt -mMsgfile}] \mbox{}

specifies the file in which the message to be 
sent is kept. 
The default is {\tt \$HOME/.vacation.msg}.



\item[{\tt -r}] \mbox{}

interval defines interval in days when not to answer 
again to the same sender. (Default is 1 day.)



\item[{\tt -d}] \mbox{}

disables the list of senders kept in the 
{\tt .vacation.pag} and {\tt .vacation.dir} files.

\end{description}


{\bf Example}

\begin{verbatim}
Subject: I am on vacation

I am on vacation until July 22. If you have something urgent, please 
contact Joe Jones (joe@blah.utoronto.ca). --john
\end{verbatim}


No message is sent if the ``user'' specified in the vacation  
command (if nothing is specified, it uses your username) does 
not appear explicitly in the {\tt To:} or {\tt Cc:} lines of the 
message, which prevents messages from being sent back to mailing lists 
and causing loops.

A list of senders is kept in the files {\tt .vacation.pag} and 
{\tt .vacation.dir} in your home directory. These are dbm database 
files. (Note: not all database systems have two files, either may be 
missing.) The vacation message is in {\tt .vacation.msg} and 
the automatic reply is activated by the {\tt .forward} (and 
saved in {\tt .vacforward}) The default vacation message is 
stored in {\tt /opt/mail/vacation.msg}

On machines running ZMailer, the {\tt name} argument 
to {\em vacation\/} is optional, and the {\tt \$USER} 
environment variable is used to determine where to look for the 
message and the list of previous recipients.

The {\tt \$SENDER} variable is checked first to determine 
the reply destination. It is normally set to the {\bf SMTP} 
{\bf MAIL FROM} address or equivalent. This is an additional 
safeguard against sending replies to mailing lists, the PostMaster 
or the mailer daemon, since standards and common sense dictate that 
it never points back to an address that could cause a loop. The 
{\tt From\_} line is used only as a last resort.




\subsection{makedb}



The way the ZMailer uses DBM entries is by using strings with 
their terminating {\tt NULL} as keys, and as data.. Thus 
the length is {\tt strlen(string)+1}, not {\tt strlen(string)} !

WARNING: Policy data parsing does use unchecked buffers!

\begin{verbatim}
Usage: makedb [-a|-p] dbtype database.name [infilename|-]
\end{verbatim}


Dbtypes are: {\tt ndbm gdbm btree bhash}

If no {\tt infilename} is defined, {\tt database.name} is assumed.

\begin{description}
\item[{\tt NDBM}] \mbox{}

appends {\tt .pag}, and {\tt .dir}
into the actual db file names.

\item[{\tt GDBM}] \mbox{}

{\bf does not} append {\tt .gdbm}
into the actual db file name.

\item[{\tt BTREE}] \mbox{}

{\bf does not} append {\tt .db}
into the actual db file name.

\item[{\tt BHASH}] \mbox{}

appends {\tt .pag}, and {\tt .dir}
into the actual db file names.

\end{description}


The {\tt -a} option is for parsing input that comes in 
{\tt aliases} format: {\tt key: data,in,single,long,line}




\subsection{dblook}



The way the ZMailer uses DBM entries is by using strings with 
their terminating {\tt NULL} as keys, and as data.. Thus the 
length is {\tt strlen(string)+1}, not {\tt strlen(string)} !

\begin{verbatim}
Usage: dblook [-dump] dbtype database.name [key]
\end{verbatim}


Dbtypes are: {\tt ndbm gdbm btree bhash}

\begin{description}
\item[{\tt NDBM}] \mbox{}

appends {\tt .pag}, and {\tt .dir} 
into the actual db file names.

\item[{\tt GDBM}] \mbox{}

{\bf does not} append {\tt .gdbm} 
into the actual db file name.

\item[{\tt BTREE}] \mbox{}

{\bf does not} append {\tt .db} 
into the actual db file name.

\item[{\tt BHASH}] \mbox{}

appends {\tt .pag}, and {\tt .dir} 
into the actual db file names.

\end{description}

%\end{multicols}


% %%%%%%%%%%%%%%%%%%%%%%%%%%%%%%%%%%%%%%%%%%%%%%%%%%%%%%%%%%%%%%%%
\clearpage
\subsection{{\tt policy-builder.sh}}

\begin{alltt}
#! /bin/sh
#
# Sample smtp-policy-db builder script.
#
# This merges following files from $MAILVAR/db/ directory:
#       smtp-policy.src
#       localnames        ('= _full_rights')
#       smtp-policy.relay ('= _full_rights')
#       smtp-policy.mx    ('relaytargets +')
#       smtp-policy.spam  ('= _bulk_mail')
#
# These all together are used to produce files:  smtp-policy.$DBEXT
# The produced database retains the first instance of any given key.

ZCONFIG=@ZMAILERCFGFILE@
. $ZCONFIG

umask 022

cd $MAILVAR/db

if [ ! -f smtp-policy.src ] ; then
        echo "No $MAILVAR/db/smtp-policy.src input file"
        exit 64 # EX_USAGE
fi

lynx -source http://www.webeasy.com:8080/spam/spam_download_table \verb/\/
   > smtp-policy.spam.new && mv smtp-policy.spam.new smtp-policy.spam

# Fork off a subshell to do it all...
(
  # The basic boilerplate
  cat smtp-policy.src

  # Localnames
  cat localnames | \verb/\/
  awk '/^#/\{next;\} NF >= 1 \{printf "%s = _full_rights\verb/\/n",$1;\}'

  # smtp-policy.relay
  # (Lists domains and networks that are allowed to use us as relay)
  if [ -f smtp-policy.relay ] ; then
    cat smtp-policy.relay | \verb/\/
    awk '/^#/\{next;\} NF >= 1 \{printf "%s = _full_rights\verb/\/n",$1;\}'
  fi

  # smtp-policy.mx
  # (Lists domains that are allowed to use us as inbound MX relay for them)
  if [ -f smtp-policy.mx ] ; then
    cat smtp-policy.mx | \verb/\/
    awk '/^#/\{next;\} NF >= 1 \{printf "%s relaytarget +\verb/\/n",$1;\}'
  fi

  # smtp-policy.spam
  # (Lists users, and domains that are known spam sources)
  # (We use file from "http://www.webeasy.com:8080/spam/spam_download_table"
  #  which is intended for QMAIL, and thus needs to be edited..)
  if [ -f smtp-policy.spam ] ; then
    cat smtp-policy.spam | tr "[A-Z]" "[a-z]" | sort | uniq | \verb/\/
    awk '/^@/\{ # Domain, but ZMailer processes them without '@'
          printf "%s = _bulk_mail\verb/\/n", substr($1,2);
          next;
        \}
        \{ # All other cases are usernames with their domains
          printf "%s = _bulk_mail\verb/\/n", $1;
        \}'
  fi

# --------- end of subshell
) | tee smtp-policy.dat | \verb/\/
$MAILBIN/makendbm -p $DBTYPE smtp-policy-new -

case $DBTYPE in
dbm)
        mv smtp-policy-new.dir  smtp-policy.dir
        mv smtp-policy-new.pag  smtp-policy.pag
        ;;
ndbm)
        mv smtp-policy-new.dir  smtp-policy.dir
        mv smtp-policy-new.pag  smtp-policy.pag
        ;;
gdbm)
        mv smtp-policy-new.gdbm smtp-policy.gdbm
        ;;
btree)
        mv smtp-policy-new.db   smtp-policy.db
        ;;
esac

exit 0
\end{alltt}


\subsection{autoanswer}



The {\em autoanswer\/} program is intended to be placed into
system global aliases database as following entry:

\begin{verbatim}
autoanswer:  "| /path/to/MAILBIN/autoanswer"
\end{verbatim}


It yields a reply message for all, except the error messages, nor
to those with {\tt X-autoanswer-loop:} header in them.

The reply sends back the original incoming message headers in the
message body along with some commentary texts.

The program is, in reality, a perl script which can easily be tuned
to local needs.

\begin{verbatim}
#!@PERL@

##########################################################################
#
# Autoanswer.pl 1.0 for ZMailer 2.99.48+
# (C) 1997 Telecom Finland
#          Valtteri Karu <valtteri.karu@tele.fi>
# 
# This program sends autoreply and the original headers to the originator 
# of the message. Version 2.99.48+ of the Zmailer is required for detecting
# possible false addresses.
#
# USAGE:
#
# Create an alias for the address to use:
# autoreply: "|/path/to/autoanswer.pl"
#
##########################################################################

$nosend = 0;
$double = 0;
$address = $ENV{'SENDER'};

if( ! -r "$ENV{'ZCONFIG'}") {
    LOG("zmailer.conf missing");
    exit 2;
}

open(ZMAILER,"< $ENV{'ZCONFIG'}" );
while(<ZMAILER>) {
    chomp;
    split(/=/);
    $ZMAILER{$_[0]}=$_[1];
}

close ZMAILER;

$logfile = $ZMAILER{'LOGDIR'} . "/autoanswer";

while (<STDIN>) {

    $text = $_;

    if (($text eq "\n") && ( $double = 1)) {
        last;
    }

    if (($text eq "\n") && ( $double = 0)) {
        $double = 1;
        next;
    }
    
    if ($text =~ "X-autoanswer-loop: ") {
        $nosend = 1;
        LOG("Looping message, sender=$address");
    }

    $double = 0;

    push(@header,$text);
}

if (($address eq '<>') || ($nosend = 0)) {
    LOG("SENDER invalid");
    exit 1;
}

$outfile = $ZMAILER{'POSTOFFICE'} . "/public/autoanswer.$$";
#$outfile = "/tmp/aa.$$";
$now = time;
$txttime = localtime(time);

open(OUT,">$outfile");
select(OUT);
print "channel error\n";
print "to $address\n";
print "env-end\n"; 
print "From: Autoreply service <postmaster>\n";
print "To: $address\n";
print "Subject: Autoreply\n";
print "X-autoanswer-loop: Megaloop \n\n";
print "      This is autoreply answer message by your request.\n\n";
print "      Original message was received at UNIX time $now;\n";
print "      which means '$txttime' in cleartext.\n\n";
print "      Headers were:\n\n";
print "--------------------------------------------------------------------\n";
print @header;
print "--------------------------------------------------------------------\n";
print "\n      Have a nice day.\n";
select(STDOUT);
close OUT;
$inode=(stat($outfile))[1];
$newfile=$ZMAILER{'POSTOFFICE'} . "/router/$inode";
rename($outfile, $newfile);
LOG("Sent to $address");
exit 0;

sub LOG {

        open(LOGf, ">>$logfile");
        $ttime = localtime(time);
        printf (LOGf "$ttime autoanswer: @_\n");
        close LOGf;
}
\end{verbatim}


\subsection{newdb}


This is elementary wrapper script building binary databases
with {\tt makedb} utility into a temporary file, and replacing
the old files with the new ones in proper order for the 
{\bf router}'s automatic source change detecting relation 
parameter {\tt -m} to work correctly.

{\bf Usage}

\begin{verbatim}
newdb /db/path/name [-u|-l] [input-file-name]
\end{verbatim}

This script uses system {\tt ZCONFIG} file to find out the desired
database type, and derives the actual database file names from the 
variable.

Suffix selection rules are:

\begin{verbatim}
dbm     .pag and .dir
ndbm    .pag and .dir
gdbm    .gdbm
btree   .db   
\end{verbatim}
