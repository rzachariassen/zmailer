\section{Sample Router Configuration Scripts}

Text to be inserted here.

The following are examples of the router configuration scripts SMTP+UUCP.cf, crossbar.cf, process.cf, and rrouter.cf.




\subsection{SMTP+UUCP.cf}

ZMailer 2 configuration file for a generic SMTP host (with UUCP links)

\begin{tscreen}
\begin{verbatim}
ZCONFIG=@ZMAILERCFGFILE@

. $ZCONFIG

PATH=.:$MAILSHARE/cf:$MAILBIN/bin ; export PATH
PS1=z$PS1
\end{verbatim}
\end{tscreen}


Configure error logging (squirrel)

\begin{tscreen}
\begin{verbatim}
squirrel -breakin
squirrel badheader
\end{verbatim}
\end{tscreen}


Domains with these toplevels will not be canonicalized via DNS lookup.
This list is from ISOC table of 16-April-95.

The quoted string {\tt "ad...zw"} should be on one line in the actual SMTP+UUCP.cf file.

\begin{tscreen}
\begin{verbatim}
toplevels="ad ae af ag ai al am an ao aq ar as at au aw az ba bb bd be bf bg bh bi bj bm bn 
bo br bs bt bv bw by bz ca cc cf cg ch ci ck cl cm cn co com cr cu cv cx cy cz de dj dk dm 
do dz ec edu ee eg eh es et fi fj fk fm fo fr ga gb gd ge gf gh gi gl gm gn gov gp gq gr gt 
gu gw gy hk hm hn hr ht hu id ie il in int io iq ir is it jm jo jp ke kg kh ki km kn kp kr 
kw ky kz la lb lc li lk lr ls lt lu lv ly ma mc md mg mh mil ml mm mn mo mp mq mr ms mt mu 
mv mw mx my mz na nc ne net nf ng ni nl no np nr nt nu nz om org pa pe pf pg ph pk pl pm pn 
pr pt pw py qa re ro ru rw sa sb sc sd se sg sh si sj sk sl sm sn so sr st sv sy sz tc td tf 
tg th tj tk tm tn to tp tr tt tv tw tz ua ug uk um us uy uz va vc ve vg vi vn vu wf ws ye yu 
za zm zr zw"
\end{verbatim}
\end{tscreen}


The transport preference order

\begin{tscreen}
\begin{verbatim}
protocols='routes smtp uucp'
\end{verbatim}
\end{tscreen}


Will the  MAILVAR/lists/listname  show out sender identity as
either:  owner-listname, or:  listname-owner?

\begin{tscreen}
\begin{verbatim}
if true ; then # Change to "false" to get "pre-owner" mode
        preowner=""
        postowner="-owner"
else
        preowner="owner-"
        postowner=""
fi
\end{verbatim}
\end{tscreen}


Does our ``local'' channel accept domain (@) at the user part?
ZMailer's mailbox does accept.  If you use something else, and
it doesn't accept, comment this away.

\begin{tscreen}
\begin{verbatim}
localdoesdomain=1
\end{verbatim}
\end{tscreen}


We may want .forward and mailing list files to be private, i.e., we ignore
the current privileges when checking the privileges of such files.
Don't add `include' to this list, since anyone can :include: any file.

\begin{tscreen}
\begin{verbatim}
private='.forward maillist'
\end{verbatim}
\end{tscreen}


Set up dependency checking.

\begin{tscreen}
\begin{verbatim}
. consist.cf
require siteinfo router crossbar process server
\end{verbatim}
\end{tscreen}


The following are standard setup files and must be loaded in this order

\begin{tscreen}
\begin{verbatim}
. standard.cf
. trusted.cf
\end{verbatim}
\end{tscreen}


Load the databases so they and the variables defined (e.g. network-specific
node names for this host) can be used in the site specific configuration.

\begin{tscreen}
\begin{verbatim}
for method in $protocols
do
        test -f $MAILSHARE/cf/i-${method}.cf && . i-${method}.cf
done

mailconf () {
        local hname

        # My official hostname
        if [ -f /bin/hostname ]; then
                rawhostname=$(/bin/hostname)
        elif [ -f /etc/sys_id ]; then
                read rawhostname < /etc/sys_id
        else
                rawhostname=$(/bin/uname -n)
        fi

        hname=$(canon $rawhostname)
\end{verbatim}
\end{tscreen}


Try to discover the organizational domain name.

\begin{tscreen}
\begin{verbatim}
        orgdomain=$hname
        sift $hname in
        $rawhostname\.(.+)
                orgdomain=\1
                ;;
        tfis
        hostname=$hname

        # This is what it will say on out mail
        mydomain=$hostname
}

orgdomains=x
: ${MAILCONF:=/etc/mail.conf}
if [ ! -r $MAILCONF ]; then
        echo "$0: missing $MAILCONF: using the following values:"
        mailconf
        echo orgdomain=$orgdomain
        echo hostname=$hostname
        echo mydomain=$mydomain
        provide siteinfo
else
        . $MAILCONF && provide siteinfo
fi
[ "$orgdomains" = x ] && orgdomains=$orgdomain
\end{verbatim}
\end{tscreen}


Set hostname to enable message-id generation and checking.

\begin{tscreen}
\begin{verbatim}
hostname $hostname

. aliases.cf
. canon.cf
. rrouter.cf
. crossbar.cf

for method in $protocols
do
        . p-${method}.cf
done

. process.cf
. server.cf

consist || exit 1
\end{verbatim}
\end{tscreen}





\subsection{Crossbar.cf}



\begin{tscreen}
\begin{verbatim}
provide crossbar
\end{verbatim}
\end{tscreen}


The crossbar function makes the policy decisions of how the instance of
a message between a particular sender and recipient should be treated.
The 'from' and 'to' parameters are quads, i.e., in the form

(channel host user attributes)

The function may modify any of these elements of both the from and to
addresses, and must select a message header address rewriting function
to be applied to this message instance.  If the return value is nil or
empty, the instance is completely ignored, to the point that if there are
no other recipients specified a complaint will be generated saying there
are {\bf no} recipients specified.

\begin{tscreen}
\begin{verbatim}
crossbar (from, to) {
        local rewrite destination tmp
\end{verbatim}
\end{tscreen}


Count them...  (in {\tt process.cf})
(we could use this as an ultimate duplicate remover too...)

\begin{tscreen}
\begin{verbatim}
        db add recipients "$(user $to)" "$(user $from)"

Intercept (drop, redirect, bounce, save) the message

        tmp=$(intercept "$(user $from)") &&
                case "$(car $tmp)" in
\end{verbatim}
\end{tscreen}

Dropping error types for from addresses is necessary to avoid mail loops.
\begin{tscreen}
\begin{verbatim}
                drop|error)
                        return ;;
                file)   LOGMSG="$LOGMSG $(car $(cdr $tmp))" ;;
                esac
\end{verbatim}
\end{tscreen}


Only intercept mail that is not from the local postmaster,
so that error messages can find their way back.

\begin{tscreen}
\begin{verbatim}
        [ "$(channel $from)" = local -a "$(user $from)" = postmaster ] ||
        tmp=$(intercept "$(user $to)") &&
                case "$(car $tmp)" in
                drop)   return ;;
                error)  setf $(channel $to) error
                        setf $(host $to) $(car $(cdr $tmp))
                        ;;
                file)   LOGMSG="$LOGMSG $(car $(cdr $tmp))" ;;
                esac
\end{verbatim}
\end{tscreen}

If we do any alias expansion from the crossbar, we should do this:
\begin{tscreen}
\begin{verbatim}
db flush expansions
\end{verbatim}
\end{tscreen}

Determine which rewrite function (for message header addresses) to use.
\begin{tscreen}
\begin{verbatim}
        case $(channel $to) in
        smtp|smtpx)
                #case "$(channel $from)" in
                #smtp|smtpx)    # Address should be forwarded the way the arrive
                #       rewrite=null ;;
                #*)     rewrite=internet ;;
                #esac
                rewrite=internet
                ;;
        error)  rewrite=null ;;
        local)  case "$(channel $from)" in
                local)  #rewrite=intramachine
                        rewrite=internet ;;
                *)      # addresses should be saved the way they arrive
                        rewrite=null ;;
                esac
                ;;
        usenet) rewrite=internet ;;
        ean)    rewrite=ean_useratdomain ;;
        *)      # This is usually UUCP or BITNET
                # We want to determine the final destination host/domain
                destination="$(uucproute "$(user $to)")"
                if [ "$(host $to)" ]; then
                        destination="$(host $to)"!"$destination"
                fi
                sift "$destination" in
                .*!([^!]+)![^!]+
                        destination="\1" ;;     # destination domain
                .*\.(bitnet|netnorth|earn|cdn)
                        rewrite=smtp_useratdomain
                        break ;;                        # reply to user@domain
                .*      rewrite=internet ; break ;;     # default sensible thing
                tfis
                ;;
        esac
\end{verbatim}
\end{tscreen}

The alias expansion might want to modify the envelope sender
of the message instance.  Here we cooperate in the scheme which
is to set the 'sender' attribute of the destination address.
\begin{tscreen}
\begin{verbatim}
        tmp="$(get $(attributes $to) sender)" && [ x"$tmp" != x ] &&
                from=(local "$tmp" "$tmp" $(attributes $from))


        case "$(channel $from)" in
        defrt1*)
                setf "$(user $from)" "$(bitnetroute "$(user $from)")"
                if [ $rewrite = internet ]; then
                        rewrite=bitnet2internet
                fi
                ;;
        esac
\end{verbatim}
\end{tscreen}

Rewrite the envelope addresses appropriately.
\begin{tscreen}
\begin{verbatim}
        case "$(channel $to)" in
#       uucp|local)
        uucp)
\end{verbatim}
\end{tscreen}

Local destination on a system that delivers in UCB Mail
compatible mail spool files means that the From\_ line
must be in all-! form, which is the same as the UUCP
transport requirement.
\begin{tscreen}
\begin{verbatim}
                setf "$(user $from)" "$(uucproute "$(user $from)")"
                setf "$(user $to)" "$(uucproute "$(user $to)")"
                sift "$(user $to)" in
                (.)!(.*)        if [ \1 = $(host $to) ]; then
                                        setf "$(user $to)" \2
                                fi
                                ;;
                tfis
                sift "$(user $to)" in
                (.)\.uucp!(.*)  setf "$(user $to)" \1!\2 ;;
                tfis
                ;;
#       smtp)
        smtp|smtpx|local|bsmtp3*)
                tmp="$(smtproute "$(user $from)")"
                sift "$tmp" in
                (@$hostname[:,].*)|([^@:,]+@$hostname)
                        break ;;
                .*
                        # tmp="@$hostname:$tmp"  # <-- that creates RFC-822
                                                 #     source-routing, AVOID!
                        tmp="$tmp"
                        ;;
                @(.+):(.+:.+)
                        tmp="@\1,\2" ; continue ;;
                tfis
                setf "$(user $from)" "$tmp"
                sift "$(user $to)" in
                (^/).*  setf "$(user $to)" "$(smtproute "$(user $to)")" ;;
                tfis

                ;;
        ean)    
                setf $(user $from) "$(ean_useratdomain "$(user $from)")"
                setf "$(user $to)" "$(ean_useratdomain "$(user $to)")"
                ;;
        usenet)
                setf $(user $from) "$(uucproute "$(user $from)")"
                sift $(user $from) in
                $hostname!.*    ;;
                .*      setf $(user $from) $hostname!$(user $from) ;;
                tfis
                # newsgroup name only
                setf "$(user $to)" $(localpart "$(user $to)")
                ;;
#       bsmtp3|bsmtp3nd)
#               setf $(user $from) "$(bitnetroute "$(user $from)")"
#               tmp="$(bitnetroute "$(user $to)")"
#               sift "$tmp" in
#               .*@([^.]).uucp
#                       tmp="$(bitnetShortroute "$(user $to)")" ;;
#               tfis
#               setf "$(user $to)" "$tmp"
#               rewrite=bitnetShortroute
#               ;;

        defrt1)
                setf $(user $from) "$(bitnetroute "$(user $from)")"
                setf $(user $to) "$(bitnetroute "$(user $to)")"
                rewrite=bitnetroute
                sift "$(user $to)" in
                (.*)[!%](.+)@(.*)
                        to=(error bitnetgw "\3" $(attributes $to))
                        rewrite=null
                        ;;
                tfis
                ;;
        esac

        #log recipient: "$(channel $to)" "$(host $to)" "$(user $to)"
        return ($rewrite $from $to)
}       # end of crossbar
\end{verbatim}
\end{tscreen}


If you want to intercept specific mail messages, this function and the
associated code in the crossbar and process functions will let you do it.  
There are three possible actions:

\begin{itemize}
\item drop      - completely ignore this address
\item error     - return the specified error message
\item file      - append the message file to the specified file
\end{itemize}


Both the file and error actions require an argument, which necessitates
the use of multiple-value return (i.e., return a list) in all cases.

If you don't want to intercept anything, this function should return failure.
The stub defined here is the usual case, you can override it in the host-
specific cf file.

\begin{tscreen}
\begin{verbatim}
intercept (address) {
#       case "$(smtp_useratdomain "$address")" in
#       *@pdq*)         return (file /var/scr/pdq) ;;
#       rayan@csri.*)   return (drop) ;;
#       bitftp*@*)      return (error bounce) ;;
#       esac

        return 1
}
\end{verbatim}
\end{tscreen}


On mail from one local user to another, we don't want to see all the
long domain name extensions.  This can cause problems with silly UAs,
if it does you can just redefine {\tt intramachine} to call 
{\tt null} in your
site or host-specific configuration files.

\begin{tscreen}
\begin{verbatim}
intramachine (address) {                # strip hostname if it came from here
        sift "$address" in
        (.*)@($hostname|$mydomain)
                address="$(condquote "\1")" ;;
        tfis
        return "$address"
}       # end of intramachine


null (address) {
        return "$address"               # surprise!
}
\end{verbatim}
\end{tscreen}


This is usually the default message-header address rewriting function.
It is responsible for hostname hiding and qualification.

\begin{tscreen}
\begin{verbatim}
internet (address) {
        address="$(canonicalize "$address")"    # Canonicalize does local
                                                # hostname hiding...
        sift "$address" in
        (.*)<@(.+)>(.*)
                        #if [ $(deliver \2) ]; then     # hostname hiding
                        #       address="\1@${mydomain}\3"
                        #       break
                        #fi
                        address="\1@\2\3" # No hostname hiding...
                        ;;
        (.*)<(.+)>(.*)  address="\1\2\3" ;;             # defocus
        [^@]+           
                # This is a local part address w/o any domains!
                address="$(condquote "$address")"
                address="$address@$mydomain"    # add our hostname
                ;;
        tfis
        return "$address"
}       # end of internet
\end{verbatim}
\end{tscreen}





\subsection{Process.cf}

This is the protocol switch function.  It keys off the form of the filename
to determine how to process a particular class of messages.  It is expected
that an internal function will be called to orchestrate the processing of
the message and enforce proper semantics.

The file argument is the name of a file in the {\tt \$POSTOFFICE/router}
directory.
\begin{tscreen}
\begin{verbatim}
process (file) {

        db flush pwuid
        db flush pwnam
        db flush fullname
        db flush hostexpansions
        db flush recipients
\end{verbatim}
\end{tscreen}

Since we cannot detect that the password database has been updated under
our feet, we flush the cached information every once in a while (in this
case, before every message).
\begin{tscreen}
\begin{verbatim}
        LOGMSG=''
\end{verbatim}
\end{tscreen}

The LOGMSG variable is used by the intercept facility (in {\tt crossbar.cf})
to make sure only a single copy of a message is saved when required.
Each sender - recipient address pair can cause an intercept which can
specify a file to save the message to.  This variable is appended to
elsewhere, and processed at the end of this function.
\begin{tscreen}
\begin{verbatim}
        case "$file" in
#       [0-9]*.x400)    x400 "$file" ;;
#       [0-9]*.uucp)    uucpfilter "$file" > /tmp/X.$$
#                       cat /tmp/X.$$ > "$file"
#                       rfc822 "$file" ;;
        [0-9]*)         rfc822 "$file" ;;
        core*)          /bin/mv "$file" ../$file.router.$$
                        return
                        ;;
        *)              /bin/mv "$file" ../postman/rtr."$file".$$
                        return
                        ;;
        esac

        [ $? ] && return 0      # Leave when they returned failure..
\end{verbatim}
\end{tscreen}

The file names in the {\tt \$POSTOFFICE/router} directory are determined by
the parameter to the mail\_open() C library routine.  This case
statement knows about the various message file types needed on your
system, and arranges appropriate processing of each.  The internal
function {\tt rfc822} expects a file name as argument, and determines the
semantics of the message and of the configuration code.  For example,
the {\tt router}, {\tt crossbar}, and {\tt header\_defer} functions have semantics only
because the {\tt rfc822} function knows about them.  There are no other
message formats supported in this distribution.
\begin{tscreen}
\begin{verbatim}
        log info: recipients $(db count recipients) $(elements $(cdr $(cdr $(cdr $(cdr $envelopeinfo))))) '
'
\end{verbatim}
\end{tscreen}

For statistics gathering we print out the envelope information property
list in its entirety, except for the file name, and the message id, both
of which were logged earlier (in C code).
\begin{tscreen}
\begin{verbatim}
        for f in $LOGMSG
        do
                { echo "==${file}==$(rfc822date)==" ;
                  /bin/cat ../queue/"$file" } >> $f && log saved "$file" in $f
        done
}
\end{verbatim}
\end{tscreen}

This does the saving of intercepted messages into archive files.










\subsection{Rrouter.cf}



Most of the address routing processing is done here.



\begin{tscreen}
\begin{verbatim}
provide rrouter
\end{verbatim}
\end{tscreen}


\begin{tscreen}
\begin{verbatim}
. fqdnalias.cf # Pick that set of tools into here!

envelopeinfo=(message-id "<$USER.interactive@$hostname>" now 0)

: ${UNRESOLVABLEACTION:='error unresolvable'}

relation -bt selfmatch selfmatch

rrouter (address, origaddr, A, plustail, domain) {
        local tmp tee didhostexpand priv nattr a
        # local seenuucp seenbitnet
        # seenuucp=false
        # seenbitnet=false
        didhostexpand="";
# echo "rrouter: address=$address, origaddr=$origaddr" >> /dev/tty

        tmp=$(fqdn_neighbour "$origaddr" "$address" $A) &&
                return $tmp

        address="$(condquote "$address")"
\end{verbatim}
\end{tscreen}

We have troublesome addresses coming here...
\begin{tscreen}
\begin{verbatim}
        #       "|pipe-program"
        #       "|quoted string"@domain
        #       "foo > faa"@domain
        #       "fii < fuu"@domain
        #       "foo @ faa"@domain
        #       "|foo @ faa"
<verb><tscreen>
and we want to do correct focusing...
<tscreen><verb>

        ssift "$address" in
        # Now make canonical
        '"'(.*)'"'<(.*)
                address="\1\2"                  # defocus
                ;;
        '"'(.*)'"'>(.*)
                address="\1\2"                  # defocus
                break ;;
        ([\'"'].*[\'"'])<(.*)
                address="\1\2" ;;               # defocus
        ([\'"'].*[\'"'])>(.*)
                address="\1\2" ;;               # defocus
        # See that it does not start with a pipe ...
#       \|.+    # Looks like a pipe... Don't mutilate it!
#               break ;;
#       '"'[|].+        # Quoted pipe??   What the ...??
#               break ;;
        tfiss

        address=$(canonicalize "$address")

        ssift "$address" in
#       <in%>(.*)
#               return (((error vms-in-pros "in%\1" $A))) ;;
#       (.*)<@(.+)\.uucp>(.*)
#               seenuucp=true
#               address="\1$plustail<@\2>\3" ;;         # fix host.uucp!route
#       (.*)<@(.+)\.(bitnet|earn|netnorth)>(.*)
#               seenbitnet=true         # Strip off the (bitnet|netnorth|earn)
#               address="\1<@\2>\4" ;;          # fix host.bitnet!route
#        handle special cases.....
#       \\(.)   return (((local - "$address" $A))) ;;
        @       # handle <> form???
                tmp=(local user postmaster $A)
                return $(routeuser $tmp "")
                ;;
\end{verbatim}
\end{tscreen}


The following two are two approaches to the same problem, generally
speaking we should use the SECOND one, but your mileage may vary...
(Problems exist when WE are the target..)

\begin{tscreen}
\begin{verbatim}
#       (.*)<@\[(.)\]>(.*)
#               address="\1$plustail<@$(gethostbyaddr \2)>\3"
#               ;;
        (.*)<@\[(.)\]>(.*)
                # numeric internet spec
                if [ $(selfmatch "\2") ]; then
                        address="\1$plustail<@>\3"
                        domain="@[\2]"
                        plustail=""
                else
                        return (((smtp "[\2]" "\1$plustail@$(gethostbyaddr \2)\3" $A)))
                fi
                ;;
\end{verbatim}
\end{tscreen}


This is the end of the $1.2.3.4$ address case...

\begin{tscreen}
\begin{verbatim}
        (.*)<@(.*)\.>(.*)
                address="\1$plustail<@\2>\3"
                plustail=""
                ;;
\end{verbatim}
\end{tscreen}


Now massage the local info.

\begin{tscreen}
\begin{verbatim}
        (.*)<@(.*)($orgdomains)>(.*)
                address="\1$plustail<@\2$orgdomain>\4"
                domain="@\2$orgdomain"
                plustail=""
                ;;
        <@(.*)>[:,](.+)@(.+)
                if [ $(deliver "\1") ]; then # Source routed to our name?
                        return $(rrouter "\2$plustail@\3" "$origaddr" $A "" "")
                fi
                ;;
        <@($orgdomains)>[:,](.+)@(.+)
                return $(rrouter "\2$plustail@\3" "$origaddr" $A "" "")
                ;;      # strip organization
        (.+)<@(.+)>(.*)
                if [ $(deliver "\2") ]; then    # Do we handle this?
                        address="\1$plustail<@>\3"
                        domain="@\2"
                        plustail=""
                elif [ "\2" = "$hostname" ]; then # Is it at local host?
                        address="\1$plustail<@>\3" # (this is a backup test)
                        domain="@\2"
                        plustail=""
                fi ;;
        <@>.(.+)        # This plustail is propably wrong...
                return $(rrouter "\1$plustail" "$origaddr" $A "" "$domain") ;;  # try after route strip
        (.+)<@> 
                if [ -z "$domain" ]; then
                        domain="$mydomain"
                fi
                return $(rrouter "\1$plustail" "$origaddr" $A "" "$domain") ;;  # strip trash & retry
        tfiss

#log "BITNET name=$bitnetname, address=$address"
        case $bitnetname in
        ?*)     tsift "$address" in
                (.*)<@(.*)\.(bitnet|netnorth|earn)>(.*)
                        address="\1<@\2>\4" ;;
                        # Strip off the (bitnet|netnorth|earn)
                tfist
                ;;
        esac
#log "BITNET name=$bitnetname, address=$address"
\end{verbatim}
\end{tscreen}


Resolve names to routes, get the actual channel name mostly from an external database.

\begin{tscreen}
\begin{verbatim}

        ssift "$address" in
        (.*)<@(.+)>(.*) 
#log "neighbourg test: domain: \2, addr: $address"
                address="\1$plustail@\2\3"
                plustail=""
\end{verbatim}
\end{tscreen}


If you want to have the SMARTHOST to pick the routing
for all non-local stuff, enable the following test case..

\begin{tscreen}
\begin{verbatim}

                # if [ "$SMARTHOST" ]; then
                #       return $(rrouter "$SMARTHOST!$(uucproute "$address")$plustail" "$origaddr" $A "" "$domain")
                # else
                #       return ((($UNRESOLVABLEACTION "$address" $A)))
                # fi


                #if [ x$seenbitnet = xtrue ]; then
                #       address="\1@\2.bitnet"
                #fi

                didhostexpand=$(hostexpansions "\2")

                for method in $protocols
                do
                        tmp=$(${method}_neighbour "\2" "$address" $A) &&
                                return $tmp
                done

                #if [ x$seenuucp = xtrue ]; then
                #       if [ "$UUCPACTION" != "" ]; then
                #               return ((($UUCPACTION "\1@\2.uucp" $A)))
                #       fi
                #       tmp=$(routes_neighbour "\2.uucp" "$address" $A) &&
                #               return $tmp
                #fi

                #if [ x$seenbitnet = xtrue ]; then
                #       if [ "$BITNETACTION" != "" ]; then
                #               return ((($BITNETACTION "\1@\2.BITNET" $A)))
                #       fi
                #fi


                if [ "$SMARTHOST" ]; then
                        return $(rrouter "$SMARTHOST!$(uucproute "$address")" "$origaddr" $A "" "$domain")
                else
                        return ((($UNRESOLVABLEACTION "$address" $A)))
                fi
                ;;

        \\(.+)  # A back-quote prefixed userid (most likely)
                return $(rrouter "\1" "$origaddr" $A "$plustail" "")
                ;;

        /.+     # file
\end{verbatim}
\end{tscreen}


Well, it could be a slash-notated X.400 address too..

\begin{tscreen}
\begin{verbatim}
                return (((local "file.$origaddr" "$address" $A)))
                ;;
        \|.+    # pipe
                return (((local "pipe.$origaddr" "$address" $A)))
                ;;
        :include:.+ # ":include:" -alias
\end{verbatim}
\end{tscreen}


We must test this here, because the file-path after
this prefix may have a dot.

\begin{tscreen}
\begin{verbatim}
                tmp=(local "$origaddr" "$address" $A)
                return $(routeuser $tmp "")
                ;;
\end{verbatim}
\end{tscreen}


Ok, from now on if we don't have a domain set, we use {\tt \$mydomain}

\begin{tscreen}
\begin{verbatim}
        .*      if [ -z "$domain" ] ; then
                        domain="@$mydomain"
                fi
                ;;
        (.+\.[^+]+)(\+.+)  # Dotfull name with a plus!
                plustail="\2"
                address="\1"
\end{verbatim}
\end{tscreen}


Fall forward for the dotfull processing.

\begin{tscreen}
\begin{verbatim}
                ;;
        .+\..+  # A dotfull name
                tmp="$(fullnamemap "$address")" && \
                        return $(rrouter "$tmp" "$origaddr" $A "$plustail" "$domain")
                if [ $(newsgroup "$address") ]; then
                        return (((usenet - "$address" $A)))
                fi
\end{verbatim}
\end{tscreen}


Okay... Not in our special fullname/newsgroup-files,
lets see if it is in the traditional one?

\begin{tscreen}
\begin{verbatim}
                if [ $(aliases "$address") ]; then
\end{verbatim}
\end{tscreen}


It can be found from the normal aliases,
run the alias processing.

\begin{tscreen}
\begin{verbatim}
                        tmp=(local "$origaddr" "$address" $A)
                        return $(routeuser $tmp "$domain")
                fi
                return (((error norealname "$address" $A)))
                return (((error nonewsgroup "$address" $A)))
                ;;

        .*      # Now all the rest of the cases..
                tmp=(local "$origaddr" "$address$plustail" $A)
                return $(routeuser $tmp "$domain")
                ;;
        tfiss
}       # end of rrouter

routes_spec (domain, address, A) {
        local tmp channel rscshost

        sift "$domain" in
#       (bsmtp3nd|bsmtp3|bitnet2|bitnet2deliver2)!(.)!(.)
        (bsmtp3nd|bsmtp3|bsmtp3nd|bsmtp3rfc|bsmtp3ndrfc)!(.)!(.)
                return (((\1 "\2@\3" "$address" $A))) ;;
        (defrt1)!(.)
                channel=\1
                rscshost=\2

                tmp="$(uucproute "$address")"
                sift "$tmp" in
                .+!([^!]+)!([^!]+)
\end{verbatim}
\end{tscreen}


We are trying to gateway through a DEFRT1 domain(!)

\begin{tscreen}
\begin{verbatim}
                        #return (((error bitnetgw "$address" $A))) ;;
\end{verbatim}
\end{tscreen}


This will usually work anyway, sigh...

\begin{tscreen}
\begin{verbatim}
                        return (((bsmtp3 "mailer@$rscshost" "\2@\1" $A))) ;;
                ([^!]+)!([^!]+)
\end{verbatim}
\end{tscreen}


The destination domain is the next hop, so we're all happy.

\begin{tscreen}
\begin{verbatim}
                        return ((($channel "\2@$rscshost" "\2@\1" $A))) ;;
                tfis
                ;;
        ignore!.*
                break
                ;;
        smtp!
                ssift "$address" in
                (.*)@(.+)
                        return (((smtp "\2" "$address" $A)))
                        ;;
                tfiss
                ;;
        dns!
                ssift "$address" in
                (.*)@(.+)
                        return (((smtp "\2" "$address" $A)))
                        ;;
                tfiss
                ;;
        (.?)!
                return ((("\1" - "$address" $A)))
                ;;
        delay!(.)
\end{verbatim}
\end{tscreen}


NB! envelope info must also be defined in interactive mode.

\begin{tscreen}
\begin{verbatim}
                tmp="$(/bin/expr $(get envelopeinfo now) + "\1")"
                return (((hold "$tmp" "$address" $A))) ;;
        (.?)!([^!]+)
                return ((("\1" "\2" "$address" $A))) ;;
        (.?)!(.+)
                # BEWARE LOOPS
                return $(rrouter "\2!$(uucproute "$address")" "$address" $A "" "$domain")
                ;;
        tfis
        return 1
}

uucproute (address) {
\end{verbatim}
\end{tscreen}


This function turns any address into a pure-! form.  It should not
call any other functions, since random other functions call it.
In particular it should not use rfc822route which itself uses
uucproute.

\begin{tscreen}
\begin{verbatim}
        sift "$address" in
        (.*)<(.*)>(.*)          address=\1\2\3 ;;               # defocus
        (.+!)@(.+)              address=\1$(uucproute "@\2") ;;
        (.+)([,:]@)(.+)         address=\1!\3 ; continue ;;
        :include:[^!]+          return $address ;;
        @(.+:)([^:]+)           address=\1$(uucproute "\2") ;;
        (.+):(.+)               address=\1!\2 ; continue ;;
\end{verbatim}
\end{tscreen}


This won't work properly for e.g. utzoo!bar@gpu.utcs.toronto.edu
because gpu.utcs also has an active uucp connection with utzoo.
It will work properly in other cases though, so if we have to guess...

\begin{tscreen}
\begin{verbatim}
        #([^!])!(.+)@(.+)       if [ $(ldotsys \1) ]; then
        #                               address=\1!\3!\2
        #                       else
        #                               address=\3!\1!\2
        #                       fi ;;
        (.+)!([^!]+)%([^!%]+)@(.+)      # route!a%b@c -> route!c!a@b
                                address=\1!\4!\2@\3 ; continue ;;
        ([^@]+)@(.+)            address=\2!\1 ;;
        @(.+)                   address=\1 ;;
        (.+)!([^!]+)[%@](.+)    address=\1!\3!\2 ;;
        tfis
        return "$address"
}       # end of uucproute
\end{verbatim}
 
\end{tscreen}
