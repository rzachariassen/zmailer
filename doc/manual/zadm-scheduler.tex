% \section{Scheduler Configuration}

The {\em scheduler} is the part of the ZMailer that manages
message processing outbound from the MTA proper.

%\begin{multicols}{2}

The major action of the Scheduler is to periodically start up Transport
Agents and tell them what to do.  This is controlled by a table in a
configuration file that is read by the Scheduler when it starts.  

Any line starting with a `{\tt \#}' character is assumed to be a comment 
line, and is ignored, as are empty lines.  All other lines must follow a rigid
format. 

The scheduler configuration file consists of a set  of
clauses. Each clause is selected by the pattern it starts
with. The patterns  for  the  clauses   are  matched,  in
sequence, with the channel/host string for each recipient
address. When a clause pattern matches  an  address,  the
parameters set in the clause will be applied to the scheduler's 
processing of that address. If the clause specifies a command,  
the clause pattern matching sequence is terminated.
Example of the clause can be seen in figure \vref{fig:adm:clause_example}.

\begin{figure*}
\begin{alltt}\hrule\medskip
local/*
        interval=10s
        expiry=3h
        # want 20 channel slots in case of blockage on one
        maxchannel=20
        # want 20 thread-ring slots
        maxring=20
        command="mailbox -8"
\medskip\hrule\end{alltt}\medskip
\caption{\label{fig:adm:clause_example}Example of {\tt scheduler.conf} clause}
\end{figure*}

A clause consists of:

\begin{itemize}
\item A selection pattern  (in  shell style) that  is
matched  against        the  channel/host  string  for an
address.
\item 0  or more  variable assignments or keywords
(described below).
\end{itemize}


If the selection pattern does not contain a '/', it is
assumed to be a channel pattern and the host  pattern  is
assumed to be the wildcard '*'.

The components of a clause are separated by whitespace.
The pattern introducing a clause must start in the first
column of a line, and the variable assignments or keywords
inside a clause must not start in the first  column of a
line. This means a clause may be written both compactly
all on one line, or spread out with an assignment or keyword per line.

If the clause is empty (i.e., consists only of a pattern),
then the contents of the next non-empty clause will be used.

The typical configuration file will contain the following clauses:
\begin{itemize}
\item a clause matching all addresses (using the pattern */*) that sets 
up default values.
\item a  clause  matching  the  local delivery channel
(usually local).
\item a clause matching the deferred    delivery  channel
(usually hold).
\item a  clause  matching  the error reporting channel
(usually error).
\item clauses specific to the other channels    known  by
the {\em router}, for example, smtp and uucp.
\end{itemize}


The  actual  names  of  these channels are completely 
controlled by the {\em router} configuration file.

Empty  lines,  and  lines   whose   first   non-whitespace
character is '\#', are ignored.

Variable values may be unquoted words or values or double 
quoted strings.  Intervals  (delta  time)  are  specified
using  a concatenation of: numbers suffixed with 's', 'm',
'h', or 'd' modifiers designating the number as a  second,
minute, hour, or day value.  For example: 1h5m20s.

The known variables and keywords, and their typical values
and semantics are:

\begin{description}

\item[\tt interval \rm(1m)] \mbox{} \\
specifies the primary retry interval, which determines how frequently a 
transport agent should be scheduled for an address. The value is a delta
time specification. This value, and the retries value mentioned below, 
are combined to determine the interval between each retry attempt.

\item[\tt idlemax] \mbox{} \\
When a transport agent runs out of jobs, they are moved
to ``idle pool'', and if a transport agent spends more than idlemax
time in there, it is terminated.

\item[\tt expiry \rm(3d)] \mbox{} \\
specifies  the  maximum age of an address in the scheduler 
queue before  a repeatedly  deferred  address  is
bounced with an expiration error.  The actual report is
produced when all addresses have been processed.

\item[\tt retries \rm(1 1 2 3 5 8 13 21 34)] \mbox{} \\
specifies the retry interval policy  of the scheduler 
for an address. The value must be a sequence of positive integers, these 
being multiples of the primary interval  before  a  retry is scheduled. 
The scheduler starts by going through the sequence as an  address  is 
repeatedly deferred. When the end of the sequence is reached, the scheduler 
will jump into the sequence at a random  spot and continue towards the end.  
This allows various retry strategies to be specified easily:

\begin{description}
\item[\rm brute force (or ``jackhammer'')]\mbox{} \\
{\tt  retries=0 }

\item[\rm constant primary interval]\mbox{}\\
{\tt  retries=1 }

\item[\rm instant backoff]\mbox{}\\
{\tt  retries="1 50 50 50 50 50 50 50 50 50 50 50 50" }

\item[\rm slow increasing (fibonacci) sequence]\mbox{}\\
{\tt  retries="1 1 2 3 5 8 13 21 34" }

\item[\rm s-curve sequence]\mbox{}\\
{\tt  retries="1 1 2 3 5 10 20 25 28 29 30" }

\item[\rm exponential sequence]\mbox{}\\
{\tt  retries="1 2 4 8 16 32 64 128 256" }

\end{description}

\item[\tt maxta \rm(0)]\mbox{}\\
if retrying an address would cause the number of simultaneously 
active transport agents to exceed the specified  value, the retry is  
postponed. The check is repeated  frequently so  the address may be retried as
soon as possible after the  scheduled  retry interval. If the value is 0, 
a value of 1000 is used instead. Keep in mind that all  running  
transport  agents  will  keep open two {\tt pipe(2)} file-handles, and thus 
system-wide  limits may force a lower maximum than 1000. On a system
with a maximum of 256 open files, this would most likely succeed at 120.

\item[\tt maxchannel \rm(0)]\mbox{}\\
if retrying an address would cause the number of 
simultaneously  active  transport agents processing mail for
the same channel to exceed  the  specified  value,  the
retry  is  postponed.  The check is repeated frequently
so the address may be retried as soon as possible after
the  scheduled  retry  interval.   If the value is 0, a
value of 1000 is used instead.

\item[\tt maxring \rm(0)]\mbox{}\\
Recipients are groupped into ``threads'',  and  similar
threads  are groupped into ``thread-rings'', where the same
transport agent can be switched over from one recipient
to another.  This defines how many transport agents can
be running at any time at the ring.

\item[\tt skew \rm(5)]\mbox{}\\
is the maximum number of retries before the retry  time
is aligned to a standard boundary (seconds since epoch,
modulo primary interval).  The lower this number (1  is
lowest), the faster the alignment is done.  The purpose
of this alignment is to ensure that eventually a single
transport agent invokation will be able to process 
destination addresses that arrived randomly at the  scheduler.

\item[\tt user \rm(root)]\mbox{}\\
is  the  user  id  of  a transport agent processing the
address.  The value is either numeric  (a  uid)  or  an
account name.

\item[\tt group \rm(daemon)]\mbox{}\\
is  the  group  id  of a transport agent processing the
address.  The value is either  numeric  (a  gid)  or a
group name.

\item[\tt command \rm(smtp -srl \$\{LOGDIR\}/smtp \$\{host\})]\mbox{}\\
is  the command line used to start a transport agent to
process the address. The program pathname is specified
relative  to the  {\tt \$MAILBIN/ta/}  directory.   The  string
``\$\{channel\}'' is replaced by the current matched  channel,
and  ``\$\{host\}'' is  replaced by the current matched host,
from the destination address, and ``\$\{LOGDIR\}'' substitutes
ZENV variable LOGDIR value there.  It is strongly 
recommended that the ``\$\{host\}'' is not to be used on a command
definition, as it limits the usability of the idled transporter.

\item[\tt bychannel]\mbox{}\\
is a keyword (with no associated value) that tells  the
scheduler  that  the transport  agent specified in the
command will only process  destination  addresses  that
match  the  first  destination  channel  it encounters.
This is automatically set when  the  string  ``\$\{channel\}''
occurs  in the command, but may also be specified manu-
ally by this keyword.  This is rarely used.

\item[\tt queueonly]\mbox{}\\
a clause with queueonly flag does not auto-start at the
arrival  of  a  message,  instead it must be started by
means of {\em smtpserver(8)} command ETRN thru an  SMTP  connection.

\end{description}

An example of full {\tt scheduler.conf} file is in
figure \vref{fig:adm:full_schedconf_sample}.

\begin{figure*}
\begin{alltt}\hrule\medskip
# Default values
*/*       interval=1m expiry=3d retries="1 1 2 3 5 8 13 21 34"
          maxring=0 maxta=0 skew=5 user=root group=daemon
# Boilerplate parameters for local delivery and service channels
local/*   interval=10s expiry=3h maxchannel=2 command=mailbox
error     interval=5m maxchannel=10 command=errormail
hold/*    interval=5m maxchannel=1 command=hold

# Miscellaneous channels supported by router configuration
smtp/*.toronto.edu
smtp/*.utoronto.ca maxchannel=10 maxring=2
          command="smtp -srl $\{LOGDIR\}/smtp"
smtp      maxchannel=10 maxring=5
          command="smtp -esrl $\{LOGDIR\}/smtp"

uucp/*    maxchannel=5 command="sm -c $\{channel\} uucp"
\medskip\hrule\end{alltt}\medskip
\caption{\label{fig:adm:full_schedconf_sample}Example of full {\tt scheduler.conf} file}
\end{figure*}


The  first  clause  (*/*)  sets up default values for all
addresses.  There is no command specification,  so  clause
matching  will  continue  after address have picked up the
parameters set here.

The third clause (error) has an implicit host wildcard  of
'*',  so  it  would  match  the same as specifying error/*
would have.

The fifth clause ({\tt smtp/*.toronto.edu}) has no further  
components so it selects the components of the following nonempty 
clause (the sixth).

Both the fifth and sixth clauses are specific  to  address
destinations  within the TORONTO.EDU and UTORONTO.CA 
organization (the two  are  parallel  domains).   At  most  10
deliveries to the smtp channel may be concurrently active,
and at most 2 for all possible hosts  within  TORONTO.EDU.
If \$host  is  mentioned in the command specification, the
transport agent will only be told about the  message  control  
files that indicate SMTP delivery to a particular
host. The actual host is picked at random from the  current  
choices, to avoid systematic errors leading to a deadlock of any queue.

%\end{multicols}
