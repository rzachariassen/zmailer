\subsection{sendmail}


This {\em sendmail\/} program is an emulation of the original
{\em sendmail\/} interface. It provides all the original options
that make sense to support in the context of ZMailer.
This is not intended to be the normal user interface to
{\em mail\/}, rather it is used by the old User Agent programs, e.g.,
{\tt mail(1)}, to submit mail. This mechanism has been superseded
by the {\tt zmailer(3)} library routines as the native
submission interface (Application Program Interface) for ZMailer.

The default action is to submit the RFC822 format mail
message expected on {\tt stdin} to the mailer, with the
addresses listed on the command line as recipients. If
there are no recipient addresses specified on the command
line, the mailer will infer them from the message header.
The sender is the account of the current {\tt userid}, 
except for {\tt root} where the preferred sender is the 
account of the current login session. The message terminates 
when a period is seen by itself on a line, or at end of file on
the input stream.

If the message submission fails immediately on the
{\tt mail\_open(3)}, the data on {\tt stdin} will be 
appended to a {\tt dead.letter} file in the submitters home directory.

{\bf Usage}

{\tt sendmail [-C configfile] [-EimqtUv] [-b[msdtip]]
[-Bbodytype] [-Nnotify] [-Rretmode]
[-Venvid] [-f address] [-F fullname]
[-r address] [-o[i{\verbar}Qpostoffice]] [address ...]}

{\bf Parameters}

\begin{description}
\item[ {\tt -bm}] \mbox{}

asks {\em sendmail\/} to deliver mail, which it does anyway. 
This option has no effect.



\item[ {\tt -bs}] \mbox{}

will start an {\em SMTP\/} server reading from {\tt stdin}. 
This causes the {\em smtpserver(8)\/} program to be executed.



\item[ {\tt -bd}] \mbox{}

starts the {\em router(8)\/} and {\em scheduler(8)\/} 
programs to emulate {\em sendmail\/}'s daemon mode. This is {\bf not} a 
recommended method to start these programs, instead use {\em zmailer(1)\/}.



\item[ {\tt -bt}] \mbox{}

runs the {\em router(8)\/} in interactive mode for testing.



\item[ {\tt -bi}] \mbox{}

runs {\em newaliases(8)\/} to rebuild the alias file database.



\item[ {\tt -bp}] \mbox{}

runs {\em mailq(1)\/} to print the mail transport queue status.



\item[ {\tt -C configfile}] \mbox{}

specifies the {\em router(8)\/} configuration file.



\item[ {\tt -E}] \mbox{}

indicates the origin of this message is an insecure 
channel. This should be used when {\em sendmail\/} is used to submit 
messages coming in from outside the local machine, to avoid security 
problems during message processing. This flag ensures the message will 
have no privileges even if the current {\tt userid} is ``trusted''.



\item[ {\tt -f address}] \mbox{}

specifies the sender address. This is the default 
originator address if there is no {\tt From:} header in the message. 
It becomes the {\tt Sender:} address otherwise. In either case if the 
current {\tt userid} is not ``trusted'' by the mailer, it is free to 
ignore both this option and any header information to ensure properly 
authenticated originator information.



\item[ {\tt -F fullname}] \mbox{}

specifies the full name of the (local) sender.



\item[ {\tt -i}] \mbox{}

tells {\em sendmail\/} to not use a period (`.') on a line 
by itself as a message terminator, only the end of file will terminate the message.



\item[ {\tt -m}] \mbox{}

asks the mailer not to ignore the originator in the 
addressee list. This is default behaviour, so this option has no effect.



\item[ {\tt -N notify}] \mbox{}

sets Delivery-Status-Notification notify parameter 
to be: {\tt NEVER}, or any combination of: {\tt SUCCESS}, 
{\tt FAILURE}, {\tt DELAY}.



\item[ {\tt -oi}] \mbox{}

is like {\tt -i}.



\item[ {\tt -oQ postoffice}] \mbox{}

specifies an alternate {\tt \$POSTOFFICE/} 
directory.



\item[ {\tt -q}] \mbox{}

asks for queue processing. This option has no effect.



\item[ {\tt -R retmode}] \mbox{}

sets Delivery-Status-Notification parameter to be 
either of: {\tt FULL}, {\tt HDRS}.



\item[ {\tt -r address}] \mbox{}

is like {\tt -f}.



\item[ {\tt -t}] \mbox{}

scans header for recipient addresses if none are specified 
on the command line. This is also the default behaviour, so this option has no effect.



\item[ {\tt -v}] \mbox{}

will report the progress of the message after it has been 
submitted. The {\em sendmail\/} process will write verbose log information 
to the {\tt stderr} stream until the {\em scheduler\/} deletes the message.



\item[ {\tt -V envid}] \mbox{}

sets Delivery-Status-Notification parameter {\tt ENVID} 
to be any arbitrary [{\tt xtext}] string.

\end{description}
